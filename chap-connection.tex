\chapter{Connection between dynamics and statics in spin glasses}
\label{chap:connection}

\section{Introduction}
\label{sec:connection-intro}

Theoretical calculations in statistical physics often involve \emph{equilibrium
  averages} over a thermodynamic ensemble, for example the canonical ensemble,
where it is assumed that the system is in equilibrium with a heat bath at fixed
temperature. In contrast, experiments and simulations usually measure
steady-state \emph{time averages}. While such \emph{static} and \emph{dynamic}
averages are usually equivalent, this equivalence can break down.

One case where this may occur is in the ordered phase of a system that exhibits
a phase transition with spontaneous symmetry breaking. For example, consider an
Ising ferromagnet in zero field. As the system is cooled below the transition
temperature, the symmetry of the paramagnetic phase is broken and subsequent
observation will find the system in one of two ordered states, ``up" or
``down," corresponding to the sign of the net magnetization.%
\footnote{%
  Here, by ``state" we mean a \emph{thermodynamic} state, corresponding to a
  probability distribution over the microscopic configurations.
}
These ``up" and ``down" states have the property that spatial correlations of
fluctuations vanish at long distances, \textit{i.e.}
\begin{equation}
  \lim_{r_{ij} \to \infty} \left( \av{S_i S_j} - \av{S_i}\av{S_j} \right) = 0.
  \label{eq:clustering-prop}
\end{equation}
\Cref{eq:clustering-prop} is called a ``clustering property" and states that
satisfy it are called ``pure" states \autocite{newman2003ordering}. The set of
pure states is ``complete" in the sense that \emph{any} thermodynamic state can
be expressed as a linear combination of pure states. More precisely,
correlation functions evaluated in an arbitrary thermodynamic state $\rho$ may
be decomposed as a convex combination%
\footnote{%
  That is, a linear combination where all coefficients are nonnegative and the
  sum of the coefficients is one.
}
of correlation functions, each evaluated in a pure state $\rho_{\alpha}$,
% TODO: reword
\begin{equation}
  \av{S_{i_1} \dots S_{i_n}}_{\rho}
  = \sum_{\alpha} W_{\alpha} \av{S_{i_1} \dots S_{i_n}}_{\rho_{\alpha}},
\end{equation}
where we say that $W_{\alpha}$ is the ``weight" of $\rho_{\alpha}$ in $\rho$.
If a state $\rho$ does not satisfy \cref{eq:clustering-prop}, more than one of
the $W_{\alpha}$ will be nonzero and we say that $\rho$ is a ``mixed" state.

Returning now to the ferromagnet, in an experiment we will find either the
``up" or ``down" state. However, the Boltzmann distribution does not exhibit
the broken symmetry of the ordered phase and includes equal contributions of
both states, giving a net magnetization of zero. Thus \cref{eq:clustering-prop}
is not satisfied and the canonical ensemble corresponds to a \emph{mixed}
state.

However, in spin glasses, which have disorder and ``frustration," the situation
is more complicated. Below the spin glass transition temperature $T_c$, a
macroscopic spin glass is not in thermal equilibrium because relaxation times
are far longer than any experimental time scale. Rather, in a typical
experiment the system is ``quenched" from a high temperature to a temperature
below $T_c$ and the subsequent dynamical evolution of the system is observed.
The state (or states) of thermal equilibrium are very complicated and are not
related to any symmetry.
% contrast with ferromagnet
As for the ferromagnet we would like to find a \emph{static} calculation which
will predict the experimental behavior, at least to some extent. Below we show
\emph{quantitatively} that the theoretical construct called the ``metastate"
\autocite{newman1997metastate,aizenman1990rounding},
combined with the technique of ``replica symmetry breaking" (RSB, see
\cref{sec:intro-rsb}), provides such a description for spin glasses, at least in
dimensions above the upper critical dimension $d_u$, where the critical
behavior is described by mean-field theory.

Pure states, those states that satisfy \cref{eq:clustering-prop}, are
convenient objects of study for several reasons. As in the case of the
ferromagnet with its ``up" and ``down" states, finding the pure states of the
ordered phase of a system provides insight into the nature of the broken
symmetry. Furthermore, \emph{pure states are observed in experiment}, while
mixed states (for example, the canonical ensemble for the ferromagnet, which
has zero magnetization even below $T_c$) are not. For these reasons we would
like to also describe spin glasses in terms of pure states. This can be done
(in principle) by taking a very large system, applying boundary conditions on
it, and studying the correlations in a relatively small window of size $W$
(with $W \ll L$) far from the boundary
\autocite{newman2003ordering,read2014short}. The assumption is that the
correlations within the window will be described by a single pure state; in a
many-states picture, different states may be observed by varying boundary
conditions at the (distant) boundary.

The question of whether there are many pure states or just one (a time-reversed
pair in the absence of a magnetic field) in spin glasses has been very
controversial.%
% TODO: ref discussion in intro
\footnote{%
  See, for example,
  \textcite{%
    parisi1980order,%
    parisi1983order,%
    fisher1987absence,%
    fisher1988equilibrium,%
    moore2011disappearance%
  }
}
If there are many, one needs to do some sort of statistical average over them,
which is called a ``metastate," for which different but equivalent formulations
have been given by \textcite{newman1997metastate} and by
\textcite{aizenman1990rounding}. The former (NS) metastate corresponds to the
distribution of states generated in a small window $W$ distant from the
boundary by varying the boundary conditions, as described above. In the latter
(AW) metastate, one considers the scale $M$, intermediate between the window
size $W$ and the system size $L$. The metastate-averaged state (MAS) is
obtained by computing correlation functions in a window in which an average is
performed not only over the spins but also over the bonds in the ``exterior"
region between $M$ and $L$. The setup is sketched in \cref{fig:metastates}.%
\footnote{%
  For more details see
  \textcite{%
    aizenman1990rounding,%
    read2014short,%
    manssen2015aging%
  }.
}
Parisi's exact solution of the infinite-range Sherrington-Kirkpatrick (SK)
model using RSB predicts many pure states (see \cref{sec:intro-rsb}) in a sense
that was later clarified by \textcite{newman1997metastate}.

\begin{figure}
  \centering
  \begin{subfigure}{0.49\textwidth}
    \centering
    \includestandalone{figures/ns-metastate}
    \subcaption{NS metastate ($W \ll L$)}
    \label{fig:ns-metastate}
  \end{subfigure}
  \begin{subfigure}{0.49\textwidth}
    \centering
    \includestandalone{figures/aw-metastate}
    \subcaption{AW metastate ($W \ll M \ll L$)}
    \label{fig:aw-metastate}
  \end{subfigure}
  \caption[
    The spin-glass metastate as defined by \textcite{newman1997metastate} and
    by \textcite{aizenman1990rounding}.
  ]
  {
    Sketch of the setups for two different (but presumably equivalent)
    formulations of the metastate given by \subref{fig:ns-metastate}
    \textcite{newman1997metastate} and by \subref{fig:aw-metastate}
    \textcite{aizenman1990rounding}.
  } \label{fig:metastates}
\end{figure}

The critical behavior of a realistic spin glass is expected to be the same as
that of the SK model in dimension $d$ greater than the upper critical
dimension, $d_u=6$. However, this does not necessarily mean that the RSB
description of the spin glass phase \emph{below} $T_c$ also applies for $d>6$.%
\footnote{%
  See
  \textcite{%
    newman1997metastate,%
    fisher1987absence,%
    fisher1988equilibrium,%
    moore2011disappearance%
  }.
}
Nonetheless, \textcite{read2014short} has computed the spatial fluctuations in a
finite-dimensional model below $T_c$, assuming mean-field (Gaussian) fluctuations,
% TODO: clarify
and the metastate description from Parisi's RSB solution of the SK model. Spin
correlations are found to fall off with a power of distance, due to averaging
over many pure states (which are unrelated by symmetry) in the metastate, \emph{i.e.}
% TODO: expand/clarify
\begin{equation}
  \av{S_i S_j}_{\mathrm{MAS}}^2 \propto r_{ij}^{-\alpha_s}\qquad
  \alpha_s = d - 4,
  \label{eq:corr-decay-static}
\end{equation}
where ``s" stands for ``static," ``MAS" stands for metastate-averaged state,
and sites $i$ and $j$ are in the window far from the boundary. For a detailed
discussion of how to do the metastate average see \textcite{read2014short}.
% TODO: should give some more detail here...

We emphasize that the calculation leading to \cref{eq:corr-decay-static}
is a \emph{static} one.
% TODO: meaning?
Is it possible to relate \cref{eq:corr-decay-static} to experiments (or
numerical simulations), which concern (non-equilibrium) dynamics?
Many simulations%
\footnote{%
  \textit{e.g.}
  \textcite{%
    manssen2015aging,%
    rieger1993nonequilibrium,%
    marinari1996numerical%
  }
}
have been carried out in which a spin glass is quenched to below $T_c$ and the
resulting dynamics analyzed. It is found that fluctuations can equilibrate (or
at least reach a steady state) on length scales smaller than a dynamical
correlation length $\xi(t)$ which is found, empirically, to grow with a power
of $t$ like
\begin{equation}
  \xi(t) \propto t^{1/z(T)},
  \label{eq:xi-scaling}
\end{equation}
where the non-equilibrium dynamical exponent $z(T)$ varies, roughly, like $1/T$
and becomes close to the critical dynamical exponent $z_c$ for $T=T_c$,
\textit{i.e.}
\begin{equation}
  1/z(T) \simeq (T/T_c) z_c.
\end{equation}
%Recently, though, \textcite{} has argued that $z(T)$ does not precisely equal
%$z_c$ in the limit $T \to T_c$.
% TODO: find ref

At distances less than $\xi(t)$ correlations are observed to fall off with a
power of distance, leading to the following scaling hypothesis,
\begin{equation}
  C_4(r_{ij},t)
  \equiv \dav{\av{S_i(t) S_j(t)}^2}
  =  r_{ij}^{-\alpha_d} f \del{\frac{r}{\xi(t)}},
  \label{eq:c4-scaling}
\end{equation}
where ``d" stands for ``dynamic." Here the square of the thermal average,
$\av{\cdots}^2$, is performed by simulation two copies of the system with the
same interactions, initialized with different random spin configurations.
Use of two copies provides an unbiased estimate of this thermal average.
% TODO: explain / ref earlier discussion
The second average, $\dav{\cdots}$, is over the bonds. We will also average over
all pairs of sites a given distance $r$ apart.

For $r_{ij} \ll \xi(t)$, $f(x)$ approaches a constant as $x \to 0$, so
% TODO: why?
\begin{equation}
  C_4(r_{ij},t) \propto r_{ij}^{-\alpha_d}
  \quad\text{($r_{ij} \ll \xi(t)$)}.
  \label{eq:corr-decay-dynamic}
\end{equation}
In the opposite limit, $r_{ij} \gg \xi(t)$, \emph{i.e.} large $x$, $f(x)$
decreases exponentially for short-range systems.
% TODO: define "short-range" system

Clearly, the nonequilibrium dynamics is generating a sampling the pure states.
To our knowledge, \textcite{white2006scenario} were the first to point out the
similarity of this sampling to the metastate average for statics. They use the
term ``maturation metastate" to describe the ensemble of states generated
dynamically on scales less than $\xi(t)$ following a quench, and ``equilibrium
metastate" for the static metastate discussed earlier. Here we will use the
terms ``dynamic" and ``static" to describe these two metastates. Subsequently
\textcite{manssen2015nonequilibrium} emphasized the similarity between the two
metastates and suggested that they might actually be equivalent, in which case
$\alpha_s$ in \cref{eq:corr-decay-static} would equal $\alpha_d$ in
\cref{eq:corr-decay-dynamic}.

The rationale behind this hypothesis is that thermal fluctuations of the spins
outside the window at a distance $\xi(t)$ and greater, which are not
equilibrated with respect to spins in the window, effectively generate a random
noise to the spins in the window which is similar to the random perturbation
coming from changing the bonds in the outer region according to the AW
metastate.

For the three-dimensional spin glass, \textcite{banos2010nature,banos2010static}
have shown that a static calculation in the zero spin overlap sector
% TODO: explain
gives a power-law decay for the spin correlations, as in
\cref{eq:corr-decay-static}, with a value of $\alpha_s$ consistent with that
obtained from dynamcs following a quench by \textcite{belletti2009depth}. These
are both numerical results. Here we consider the mean-field regime, $d>6$,
because there is an exact \emph{analytic} result in RSB theory, $\alpha_s=d-4$,
with which we can compare our numerical results.


\section{Model}

Unfortunately, it is difficult to carry out Monte Carlo simulations of spin
glasses in in six dimensions (see the discussion in
\cref{sec:nonextensive-motivation}). Instead, we study the \emph{diluted}
one-dimensional models with long-range interactions described in
\cref{sec:nonextensive-models}, which we briefly review here. The
one-dimensional diluted model is described by the Hamiltonian
\begin{equation}
  \ham = -\sum_{i,j} J_{ij} S_i S_j,
\end{equation}
where the sites $i\in\cbr{1,2,\dots,N}$ lie on a one-dimensional ring with
periodic boundary conditions, as shown in \cref{fig:1dlr-chord}. The variables
$S_i = \pm 1$ are Ising spins, and the interactions $J_{ij}$ are independent
random variables with a distribution satisfying
\begin{equation}
  \dav{J_{ij}}=0,\qquad
  \dav{J_{ij}^2} \propto R_{ij}^{-2\sigma},
  \label{eq:J-mean-variance-lr-2}
\end{equation}
where $R_{ij}$ is taken to be the chord distance between sites $i$ and $j$, see
\cref{fig:1dlr-chord}. We vary the parameter $\sigma$ to control the range of
the interactions.

The \emph{diluted} model corresponds to a particular choice of the distribution
$P(J_{ij})$ that satisfies \cref{eq:J-mean-variance-lr-2} while allowing for
efficient simulation, namely
\begin{equation}
  P(J_{ij})
  = \del{1 - p_{ij}} \delta \del{J_{ij}}
  + p_{ij} \frac{1}{\sqrt{2\pi}} e^{-J_{ij}^2/2},
  \label{eq:dilute-dist-2}
\end{equation}
where $p_{ij} \propto R_{ij}^{-\sigma}$ at large distance and the constant of
proportionality is chosen to fix the mean number of neighbors $z_b$ ($=6$ in
this work). See \cref{sec:nonextensive-models} for a detailed discussion of the
diluted model and an algorithm to sample from \cref{eq:dilute-dist-2}.

As discussed in \cref{sec:nonextensive-motivation}, varying the parameter
$\sigma$ is argued to be analogous to changing the dimension $d$ of a
short-range model. In the mean-field regime, $d>d_u=6$ for the short-range
model, a precise connection can be given between $\sigma$ and an equivalent
$d$, namely
\begin{equation}
  d = \frac{2}{2\sigma - 1}
  \label{eq:d-sigma}
\end{equation}
(see \cref{sec:nonextensive-motivation}), and thus, for the long-range model,
the mean-field regime is $1/2 < \sigma < 2/3$.

The connection between critical exponents of the short-range and corresponding
long-range models has been discussed systematically by
\textcite{banos2012correspondence}, who note that an exponent of the
short-range model in $d$ dimensions is $d$ times the corresponding exponent of
the equivalent one-dimensional long-range model.
% TODO: explain
Thus, to get the exponent $\alpha_s=d-4$ in the static metastate for the
long-range model we divide by $d$ and, since we will work in the mean-field
regime, use \cref{eq:d-sigma} to relate $d$ to $\sigma$. This gives
\begin{equation}
  \alpha_s = 3-4\sigma
  \quad\text{(long-range model).}
  \label{eq:alpha-rsb-lr}
\end{equation}

In this work we focus on a single value of $\sigma$ in the mean-field regime,
$\sigma=5/8$, which corresponds to $d=8$ according to \cref{eq:d-sigma}. Using
standard finite-size scaling analysis (see \cref{sec:numerical-fss}) we find
that $T_c=1.85(2)$ for this model with $z_b=6$.
% TODO: define z_b
Here we need to work \emph{well} below $T_c$ so that our data is characteristic
of the ordered phase and does not also incorporate critical fluctuations.
Thus we take $T = 0.4 T_c = 0.74$ for the simulations.


\section{Method}

We quench the system from infinite temperature to $T=0.74$ at time $t=0$ and
follow the evolution of the system using Monte Carlo simulation with only local
(\textit{e.g.} \emph{not} replica exchange) updates. We measure spin
correlations, averaging them for times between $2^{k-1}$ and $2^k$, for integer
$k$ up to the maximum value. For the largest sizes this was $k=14$. We find
that finite-size effects are very large and we need to study enormously large
sizes. We therefore take a range of sizes which also increases geometrically,
$N=2^{\ell}$ up to $\ell=26$. We also average over about 1000 samples (the
precise number depending on size).
% TODO: table of simulation params


\section{Results}

\Cref{fig:c4-vs-r-sizes} shows our data for the correlation function $C_4(r,t)$,
defined in \cref{eq:c4-scaling}, as a function of $r\equiv\abs{i-j}$ at
$t=2^{14}$ for different sizes. Despite the strong finite-size effects, the data
seems to have saturated for the largest sizes, at least for the range of distance
presented.
\begin{figure}
  \centering
  \includestandalone{figures/c4-vs-r-sizes}
  \caption[
    Data for the correlation function averaged between $t=2^{13}$ and $2^{14}$
    for different sizes of the one-dimensional long-range diluted spin glass
    with $\sigma=5/8$ at $T=0.4T_c$.
  ]
  {
    Data for the correlation function $C_4(r,t)$, defined in \cref{eq:c4-scaling},
    as a function of $r$ for the range of sizes studied. The data is averaged
    between $t=2^{13}$ and $2^{14}$.
  } \label{fig:c4-vs-r-sizes}
\end{figure}

Having established that the largest size, $N=2^{26}$, is large enough to
eliminate finite-size effects for the range of $r$ and $t$ considered, we now
discuss the data for this size in detail. \Cref{fig:c4-vs-r-times} shows data
for $C_4(r,t)$ at different times as a function of $r$. It is expected to have
the scaling form shown in \cref{eq:c4-scaling}. For short-range models, the
scaling function $f(x)$ decays exponentially at large $x$ because the
correlation function falls off very rapidly once $r$ is greater than the
dynamical correlation length. However, in the present model we have
interactions of arbitrarily long range which give a ``direct" contribution to
the correlation function at large distances. Since $C_4(r,t)$ involves the
square of the spin-spin correlation function, and is averaged over the
interactions, the direct contribution should be proportional to
$\dav{J_{ij}^2}$, which, according to \cref{eq:J-mean-variance-lr-2}, is
proportional to $r^{-2\sigma}$ ($=r^{-5/4}$ for $\sigma=5/8$). The data in
\cref{fig:c4-vs-r-times} follow this behavior for short times and large
distances, see the dotted line.

By contrast, at small $r$ and large $t$, where $r \ll \xi(t)$, the data for
different times collapse and are consistent with a decay proportional to
$r^{-(3-4\sigma)}$ ($=r^{-1/2}$ for $\sigma=5/8$), see the dashed line in
\cref{fig:c4-vs-r-times}. To better estimate the slope at large $t$ and small
$r$ we plot in the inset to \cref{fig:c4-scaling} the ``effective" exponent
$\alpha_{\mathrm{eff}}$, the slope of the data in \cref{fig:c4-vs-r-times}, as
a function of $r$ for different times. The curves are quadratic fits for
intermediate $r$ ($7 \leq r \leq 255$). The intercepts of the fits approach
$-1/2$ for $r \to 0$ at large $t$. Thus, according to
\cref{eq:corr-decay-dynamic}, we have $\alpha_d=3-4\sigma$ (or at least very
close to it). However, this is precisely equal to $\alpha_s$, the corresponding
exponent from the \emph{static} metastate according to RSB theory as shown in
\cref{eq:alpha-rsb-lr}. Thus we see that, in the mean-field regime, the static
and dynamic metastates appear to agree and the description appears to be that
of RSB. The latter is in agreement with several other studies
\autocite{moore2011disappearance,katzgraber2005probing} and is of course also
implied by those, such as \textcite{banos2010nature,banos2010static}, which
argue that RSB holds even below six dimensions.
\begin{figure}
  \centering
  \includestandalone{figures/c4-vs-r-times}
  \caption[
    Data for the correlation function at different times of the one-dimensional
    long-range diluted spin glass with $\sigma=5/8$ at $T=0.4T_c$.
  ]
  {
    Data for the correlation function for the largest size $N=2^{26}$ as a
    function of $r$ for different times. A gradual crossover can be seen
    between two power laws. At long times and short distances $C_4(r,t) \propto
    1/r^{\alpha_d}$ with $\alpha_d = 3 - 4\sigma$ (dashed line); at short times
    and long distances $C_4(r,t) \propto 1/r^{-2\sigma}$ (dotted line) which is
    just the average of the square of the interactions $J_{ij}$.
  } \label{fig:c4-vs-r-times}
\end{figure}

The main part of \cref{fig:c4-scaling} shows a scaling plot of our data for the
largest size according to \cref{eq:c4-scaling,eq:xi-scaling}. The data scale
well with $z(T)=1.4$ and, including estimated error bars, we have the result
$z(0.4 T_c)=1.4(2)$ for the dynamical exponent describing the growth of
nonequilibrium correlations following a quench. For short-range models it is
found empirically%
\footnote{%
  See
  \textcite{%
    manssen2015aging,%
    rieger1993nonequilibrium,%
    % TODO: missing ref
    marinari1996numerical,%
    yoshino2002extended%
  }.
}
that $1/z(T) \propto T$. If we assume the same here then $z(T_c)=0.56(8)$.
Furthermore, still for short-range models it is also found that $z(T_c)$,
obtained from nonequilibrium data, is equal to (or at least close to)
the equilibrium dynamical exponent $z_c$. We therefore take $z_c=0.56(8)$
for our long-range model. To translate this into the exponent for the
equivalent short-range model, we multiply by $d$ ($=8$), as discussed above,
so our estimate for the critical dynamical exponent of the $d=8$ short-range
spin glass is $z_c=4.5(6)$ ($d=8$). This model is in the mean-field regime
($d>6$) for which the dynamical exponent is found to be $z_c=4$
\autocite{zippelius1984critical}. Our result is consistent with this.

\begin{figure}
  \centering
  \includestandalone{figures/c4-scaling}
  \caption[
    Scaling plot of data for the dynamical correlation function $C_4(r,t)$ at
    different times for the one-dimensional long-range diluted spin glass with
    $\sigma=0.74$ at $T=0.4T_c$.
  ]
  {
    Scaling plot of the data for the largest size $N=2^{26}$ at $T=0.74$
    according to \cref{eq:c4-scaling}. The inset shows the effective exponent
    $\alpha_{\mathrm{eff}}$, the slope of the curves in
    \cref{fig:c4-vs-r-times}, as a function of $r$. The curves in the inset are
    quadratic fits to the data for intermediate $r$, $7 \leq r \leq 255$. The
    intercepts of the fits approach $-0.5$ at long times.
  }
  \label{fig:c4-scaling}
\end{figure}


\section{Conclusion}

We have shown quantitatively that the nonequilibrium dynamics following a
quench of a model which is a proxy for a short-range spin glass in dimension
$d>6$ is given, in the steady-state regime where the distance is less than the
nonequilibrium correlation length, by the \emph{analytic} result for the
\emph{static} metastate calculated according to RSB theory. This suggest that
(i) RSB theory applies to spin glasses above the upper critical dimension,
$d_u=6$, and (ii) the dynamic and static metastates are equivalent (at least in
this region).
