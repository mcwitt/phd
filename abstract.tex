\begin{abstract}
  \DoubleSpacing
  Despite decades of effort, our understanding of low-temperature phase of spin
  glass models with short-range interactions remains incomplete. Replica
  symmetry breaking (RSB) theory, based on the solution of the
  Sherrington-Kirkpatrick mean-field model, predicts many pure states;
  meanwhile, competing theories of short-range systems, such as the droplet
  picture, predict a single pair of pure states related by time-reversal
  symmetry, analogously to the ferromagnet. Since RSB certainly holds for the
  mean-field (infinite-range) model, it is interesting to study short-range
  models in high dimensions to observe whether RSB also holds here; however,
  computer simulations of short-range models in high dimensions are difficult
  because the number of spins to equilibrate grows so rapidly with the linear
  size of the system.

  A relatively recent idea which has been fruitful is to instead study
  one-dimensional models with long-range (power-law) interactions, which are
  argued to have the same critical behavior as corresponding short-range models
  in high dimensions, but for which simulations for a range of sizes (crucial
  for finite-size scaling analysis) are feasible. For these one-dimensional
  long-range (1DLR) models, we fill in a previously unexplored region of
  parameter space where the interactions become sufficiently long-range that
  they must be rescaled with the system size to maintain the thermodynamic
  limit. We find strong evidence that detailed behavior of the 1DLR models
  everywhere in this ``nonextensive regime" is identical to that of the
  Sherrington-Kirkpatrick model, lending support to a recent conjecture.

  In an attempt to distinguish the RSB and droplet pictures, we study
  recently-proposed observables based on the statistics of individual disorder
  samples, rather than simply averaging over the disorder as is most frequently
  done in previous studies. We compare Monte Carlo results for 1DLR models
  which are proxies for short-range models in 3, 4, and 10 dimensions with
  previously-obtained data for the 3D and 4D short-range models and the SK
  model. For one statistic, which is expected to sharply distinguish between
  the two pictures in the thermodynamic limit, we find that larger system sizes
  than those currently feasible to simulate are needed to obtain an unambiguous
  result. We also find that two other recently-proposed statistics, the median
  of the cumulative overlap distribution and the ``typical" overlap
  distribution, are not particularly helpful in distinguishing between the RSB
  and droplet pictures.

  If there are many pure states in the spin-glass phase, we need to carry out
  some sort of statistical average over them to obtain the thermodynamics. One
  such prescription for doing this is called the ``metastate." Motivated by
  similarities between the average over pure states specified by the metastate
  theory and that presumably generated by the nonequilibrium dynamics, we study
  a 1DLR model which is a proxy for a short-range model in $d=8$ dimensions and
  measure the evolution of dynamical correlations. We find that the spatial
  decay of the correlations at distances less than the dynamical correlation
  length $\xi(t)$ agrees quantitatively with the predictions of the metastate
  theory, evaluated according to the RSB picture. We also compute the dynamic
  exponent defined by $\xi(t) \propto t^{1/z(T)}$ and find that it is
  compatible with the mean-field value of the critical dynamical exponent for
  short-range spin glasses.

  Finally, we present a unified view of finite-size scaling (FSS) in dimensions
  $d$ above the upper critical dimension $d_u$, for both free and periodic
  boundary conditions. For $d>d_u$, a dangerous irrelevant variable is
  responsible for both the violation of hyperscaling and the violation of
  ``standard" FSS. We find that the modified hyperscaling proposed to allow for
  this applies only to $\vec{k}=\vec{0}$ fluctuations, while standard FSS
  applies to $\vec{k}\neq\vec{0}$ fluctuations. Hence the exponent $\eta$
  describing the power-law decay of correlations at criticality is
  unambiguously $\eta=0$. With free boundary conditions, the finite-size
  ``shift" is greater than the rounding. Nonetheless, using $T-T_L$, where
  $T_L$ is the finite-size pseudocritical temperature, as the scaling variable,
  the data do collapse onto a scaling form that includes the behavior both at
  $T_L$, where the susceptibility $\chi$ diverges like $L^{d/2}$, and the bulk
  $T_c$, where it diverges like $L^2$. We support these claims with data from
  large-scale simulations of the five-dimensional Ising model.
\end{abstract}
