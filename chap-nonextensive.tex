\chapter{Nonextensive spin glasses}
\label{chap:nonextensive}

\section{Motivation}
\label{sec:nonextensive-motivation}

In the study of phase transitions, it is often desirable to study models in a
range of spatial dimensions between the lower critical dimension $d_l$, below
which there is no finite-temperature transition, and the upper critical
dimension $d_u$, above which the critical behavior is mean-field-like, meaning
that the critical exponents are determined by mean-field theory.

But numerical simulations with large $d$ are difficult because the number of
spins, and hence the time and space required for the simulation, grows rapidly
with size as $N=L^d$. For glassy systems, situation is especially difficult
because the characteristic time scales, and thus the amount of simulation time
required to equilibrate the system, also grows rapidly with size. Presently the
best-known Monte Carlo methods for glassy systems are able to equilibrate up to
about $10^4$ spins in a reasonable amount a computer time, which limits system
sizes at $d_u=6$ to $L \lesssim 10$. For such small sizes finite-size scaling
analysis is difficult or impossible, not only because we lose the ability to
study a \emph{range} of $L$ from which to infer the bulk behavior, but also
because corrections to scaling become significant for small $L$.


\subsection{Long-range interactions}

Some of these difficulties can be avoided by instead studying one-dimensional
models with long-range interactions, as proxies for higher-dimensional models
with short-range interactions. For example, consider the Ising
chain with only nearest-neighbor interactions, which is below the lower
critical dimension $d_l=2$ for the Ising ferromagnetic transition and therefore
does not have a finite-temperature phase transition. However, with the addition
of long-range interactions falling off as $r^{-\sigma}$, there \emph{is} a
finite-temperature transition for $\sigma \leq 2$
 and, furthermore, the critical behavior is mean-field-like for $\sigma < 4/3$
% TODO: strictly < or <=?
 \autocite{dyson1969existence}.
Thus we establish a correspondence between the lower- and upper-critical
dimensions $d_l=2$ and $d_u=4$ of the short range model, and the lower- and
upper-critical \emph{ranges} $\sigma_l=2$ and $\sigma_u=4/3$ of the chain with
long-range interactions. The \emph{continuous} parameter $\sigma$ of the chain
plays a role analagous to the dimensionality $d$ of the short-range model
\autocite{katzgraber2003monte}, which is plausible because both parameters
essentially control the degree of the coupling between spins.

Note that increasing $\sigma$ \emph{decreases} the range of the interactions,
and thus plays a similar role of reducing the connectivity of the spins as
decreasing $d$. The extreme case $\sigma=0$ is the ``infinite-range" model
which is solved exactly by mean-field theory, while the opposite extreme
$\sigma\to\infty$ is the short-range model. We define the ``effective
dimension" $d_{\mathrm{eff}}(\sigma)$ such that a one-dimensional model with
long-range interactions falling off as $r^{-\sigma}$ has the same critical
behavior as the equivalent short-range model with $d=d_{\mathrm{eff}}$. In the
mean-field regime (\emph{e.g.} $\sigma<4/3$ for the Ising ferromagnet) we can
derive $d_{\mathrm{eff}}(\sigma)$ in closed form by equating the singular parts
of the free energy density for the two models \autocite{larson2010numerical}.
This is discussed in \cref{sec:nonextensive-lrsg}, where we compute
$d_{\mathrm{eff}}(\sigma)$ for the one-dimensional spin glass with long-range
interactions.
% TODO: comment about non-MF

One-dimensional models with long range interactions have the important
advantage that, since $N=L$, there is no difficulty in studying many different
sizes for finite-size scaling (FSS) analysis. But although we can feasibly
study large $L$ using this method, it is important to note that finite-size
effects are not reduced---we have traded small $L$ and short-range interactions
for large $L$ and long-range interactions, so that the ratio of the length
scale of fluctuations to the system size is essentially unchanged.
% TODO: another advantage: simpler geometry than d-dimensional hypercubic?


\subsection{Nonextensive regime}

Considering the general case of a $d$-dimensional model with long-range
interactions falling off as $r^{-\sigma}$, we note that, if $0 \leq \sigma \leq
d$, the free energy is nonextensive, \emph{i.e.} grows faster than $N$ in the
thermodynamic limit. We call this region of parameter space the
\emph{nonextensive regime}.

For example, the mean-field transition temperature for an Ising ferromagnet
with long-range interactions, given by
\begin{equation}
  \tcmf
  \equiv \sum_j J_{ij}
  = c \sum_{j \neq i} r_{ij}^{-\sigma},
  \label{eq:TcMF-ferro}
\end{equation}
behaves asymptotically like
% TODO: consider specializing discussion to d=1
\begin{equation}
  \tcmf/J
  \sim \int \dif^d r \, r^{-\sigma} \propto
  \begin{cases}
    1 + a_1 L^{d-\sigma} & \text{if } d \neq \sigma,\\
    1 + a_2 \log L       & \text{if } d = \sigma,
  \end{cases}
\end{equation}
where $a_1$ and $a_2$ are constants. Therefore if we take $c$ in
\cref{eq:TcMF-ferro} to be independent of $N$, the mean-field transition
temperature diverges in the thermodynamic limit unless $\sigma > d$.
Nevertheless it is possible to define a model with $\sigma \leq d$ and a
sensible thermodynamic limit provided we also take $c \to 0$ in such a way that
the mean-field transition temperature remains finite \autocite{cannas1996long}.
For $\sigma < d$ we require
\begin{equation}
  c = c(\sigma, N) \propto N^{\sigma/d - 1}.
\end{equation}
To fix the constant of proportionality we adopt the convention $\tcmf=1$, which
for the ferromagnet gives
\begin{equation}
  J_{ij} = \frac{c(\sigma, N)}{r_{ij}^{\sigma}},
  \qquad
  c(\sigma, N) = \del{ \sum_{j \neq i} r_{ij}^{-\sigma} }^{-1}.
  \label{eq:J-scaling-ferro}
\end{equation}

The nonextensive regime for \emph{ferromagnets} has already been explored. In
one of the earliest studies
% TODO: the earliest?
of ferromagnets with long-range interactions, \textcite{hiley1965ising} showed
that the transition temperature approaches the mean-field prediction in the
limit $\sigma \to d$ from above. This led \textcite{cannas1996long} to
conjecture that, for a ferromagnetic model in which the interactions are
appropriately scaled in the nonextensive regime, the mean-field transition
temperature is exact not only for $\sigma=0$ and $\sigma \to d^+$, but in the
entire range $0 \leq \sigma \leq d$. This conjecture was subsequently verified
numerically \autocite{cannas2000evidence,campa2000canonical}.
% TODO: other refs?

It is interesting to ask if the same is true for spin glasses.
\textcite{mori2011instability} argues that this is the case; that is, the
behavior throughout nonextensive regime is identical to that of the mean-field
model of spin glasses, provided the couplings are appropriately scaled so that
the mean-field transition temperature is independent of system size. Our
motivation will be to test this conjecture using Monte Carlo simulation. First,
we introduce a long-range spin glass model and define the corresponding
nonextensive regime.


\subsection{Spin glasses with long-range interactions}
\label{sec:nonextensive-lrsg}

Long-range spin glass models may be defined by taking the variance of the
interactions to fall off with distance. Such a model was first studied by
\textcite{kotliar1983one}, who considered a one-dimensional Ising chain with
the Hamiltonian
\begin{equation}
    \ham = -\frac{1}{2} \sum_{i,j} J_{ij} S_i S_j,
    \qquad
    J_{ij} = \epsilon_{ij}/r_{ij}^\sigma,
\end{equation}
where $\epsilon_{ij}$ are independent, identically-distributed Gaussian random
variables. For this model they found a finite-temperature phase transition for
$\sigma<1$ and mean-field behavior for $1/2 < \sigma < 2/3$; that is,
$\sigma_l=1$ and $\sigma_u=2/3$.

The extreme case $\sigma=0$ corresponds to the Sherrington-Kirkpatrick
mean-field model (see \cref{sec:intro-sk}), for which the transition
temperature is exactly
\begin{equation}
  \tcmfsq \equiv \sum_j \dav{J_{ij}^2} = c \sum_{j \neq i} r_{ij}^{-2\sigma},
\end{equation}
where $\dav{\cdots}$ denotes an average over bonds. Compared with
\cref{eq:TcMF-ferro} for the ferromagnet, note that $J_{ij}$ ($\propto
1/r_{ij}^{\sigma}$), which is zero on average for a spin glass, is replaced by
the variance $\dav{J_{ij}^2}$ ($\propto 1/r_{ij}^{2 \sigma}$).
% TODO: above, define "bond average"
% TODO: ref sk model section in intro

For $0 \leq \sigma \leq 1/2$ the free energy diverges in the thermodynamic
limit unless the interactions are scaled by an inverse power of the system
size.

Again adopting the convention $\tcmf=1$, the analog of
\cref{eq:J-scaling-ferro} for the spin glass is
\begin{equation}
  \dav{J_{ij}^2} = \frac{c(\sigma, N)}{r_{ij}^{2\sigma}},
  \qquad
  c(\sigma, N) = \del{\sum_{j \neq i} r_{ij}^{-2\sigma}}^{-1}.
\end{equation}

In the mean-field regime ($\sigma<4/3$) we can make a precise connection
between the power $\sigma$ of the one-dimensional long-range (1D LR) model and
the dimensionality $d$ of the short-range (SR) model by equating the singular
parts of the free energy density for the two models
\autocite{larson2010numerical}. The singular part of the free energy density
scales with $L$ like
\begin{equation}
  \frac{1}{L^d} \widetilde{f} \del{L^{y_T} t,\, L^{y_H} h,\, L^{y_u} u}
\end{equation}
where $\widetilde{f}$ is a scaling function, $t=(T-T_c)/T_c$ is the reduced
temperature, and $h$ is the magnetic field. Eliminating $L$ (using $N=L$ for
the LR model and $N=L^d$ for the SR model) and equating the resulting
expressions for the two models gives the following relationship between the LR
and SR exponents:
\begin{equation}
  y^{\mathrm{LR}}(\sigma) = y^{\mathrm{SR}}(d)/d.
\end{equation}
In the mean field regime, where the FSS exponents for the SR and LR models are
given by
\begin{equation*}
  \begin{split}
    y_T^{\mathrm{SR}} &= 2 \\
    y_H^{\mathrm{SR}} &= (d+2)/2 \\
    y_u^{\mathrm{SR}} &= (6-d)/2
  \end{split}
  \qquad
  \begin{split}
    y_T^{\mathrm{LR}} &= 2\sigma - 1 \\
    y_H^{\mathrm{LR}} &= \sigma \\
    y_u^{\mathrm{LR}} &= 3\sigma - 2
  \end{split}
\end{equation*}
\autocite{harris1976critical,kotliar1983one}, we obtain the following
relationship between the power $\sigma$ of the 1D LR model and the
corresponding effective dimension,
\begin{equation}
  d_{\mathrm{eff}} = \frac{2}{2\sigma - 1},
\end{equation}
consistently for each pair of exponents.

\begin{figure}
  \centering
  \includestandalone{figures/1dlr-phases}
  \caption[
    Schematic phase diagram of the one-dimensional Ising spin glass with
    power-law interactions.
  ]
  {
    \emph{Adapted from \textcite{katzgraber2003monte}}.
    Schematic phase diagram in the $d$-$\sigma$ plane for the one-dimensional
    Ising spin glass with power-law interactions, following
    \textcite{fisher1988equilibrium}. In the shaded regions there is a
    transition with $T_c>0$. The line $d=2\sigma$ separates the nonextensive
    (NE) region from the mean-field (MF) region. The critical exponents in both
    the NE and MF regions are those of mean-field theory. The line
    $d=3\sigma/2$ (red) separates the mean-field region from the long-range
    (LR) region where the critical exponents differ from mean-field theory, but
    the dominant contribution to the energy comes from the long-range
    interactions. On the far right are the short-range (SR) regions where the
    energy is dominated by the short-range interactions. These are separated
    from the long-range regions by the curve
    $\theta_{\mathrm{LR}}=\theta_{\mathrm{SR}}$ (red), where
    $\theta_{\mathrm{LR}}$ and $\theta_{\mathrm{SR}}$ are the stiffness
    exponents corresponding to the long- and short-range interactions
    respectively, see \cref{sec:intro-theories}.
  }
\end{figure}

\begin{figure}
  \centering
  \includestandalone{figures/1dlr-chord}
  \caption[
    Representation of the one-dimensional long-range Ising spin glass
    with periodic boundary conditions.
  ]
  {
    Representation of the (undiluted) one-dimensional long-range spin glass
    with $r_{ij}$ defined as the chord distance when the spins are arranged on
    a ring. The edge widths are proportional to the variance of the $J_{ij}$
    with $\sigma=0.375$.
  } \label{fig:1dlr-chord}
\end{figure}


\section{Models}
\label{sec:nonextensive-models}

We study the Hamiltonian
\begin{equation}
    \ham = -\frac{1}{2} \sum_{i,j} J_{ij} S_i S_j
\end{equation}
where the couplings $J_{ij}$ are sampled from a distribution with mean zero and
variance that falls off with a power of the distance $r_{ij}$,
\begin{equation}
  \dav{J_{ij}} = 0,
  \qquad
  \dav{J_{ij}^2} \propto r_{ij}^{-2\sigma},
  \label{eq:J-mean-variance-lr}
\end{equation}
where to define $r_{ij}$ we put the spins on a ring and use the chord distance,
\begin{equation}
  r_{ij} = \frac{L}{\pi} \sin\del{\frac{\pi\abs{i-j}}{L}}
  \label{eq:chord-distance}
\end{equation}
(see \cref{fig:1dlr-chord}).

We choose a particular distribution $P(J_{ij})$, introduced by
\textcite{leuzzi2008dilute}, that satisfies \cref{eq:J-mean-variance-lr}
asymptotically while allowing for efficient computer simulation. In the
resulting model, which we refer to as the \emph{diluted model}, the interaction
matrix $J_{ij}$ is sparse, with the mean number of neighbors of a given spin
fixed to an arbitrary constant $z$ (here we choose $z=6$). The nonzero elements
of $J_{ij}$ are drawn from a Gaussian distribution with zero mean and unit
variance. The second part of \cref{eq:J-mean-variance-lr} is then satisfied by
letting the \emph{probability} of an interaction between spins $i$ and $j$ fall
off with distance like $r^{-2\sigma}$. That is, the interactions are
distributed according to
\begin{equation}
  P(J_{ij})
  = \del{1 - p_{ij}} \delta \del{J_{ij}}
  + p_{ij} \frac{1}{\sqrt{2\pi}} e^{-J_{ij}^2/2},
  \label{eq:dilute-dist}
\end{equation}
where $p_{ij} \propto r_{ij}^{-2\sigma}$.

To sample from this distribution we use the following algorithm. Choose a site
$i$ at random from a uniform distribution. Then choose a site $j$ with
probability $\widetilde{p}_{ij}=A/r_{ij}^{2\sigma}$, where $A$ is determined by
normalization. If there is not already a bond between $i$ and $j$, take
$J_{ij}$ from a Gaussian distribution with zero mean and unit variance.%
\footnote{
  Note that if $\widetilde{p}_{ij} \ll 1$, then $p_{ij}$ in
  \cref{eq:dilute-dist} is given by $p_{ij} = z \widetilde{p}_{ij}$. Otherwise
  there will be corrections due to the rejection of bonds when there is already
  a bond between $i$ and $j$.
}
Repeat until $Nz/2$ bonds have been generated in total, at which point the
number of neighbors of a given spin has a Poisson distribution with mean $z$.
% TODO: standardize bonds/interactions/connections
% TODO: algorithm pseudocode?
% TODO: ref discussion of diluted model

The transition temperature for the diluted model with $\sigma=0$ was shown by
\textcite{viana1985phase} to be given by the solution of
\begin{equation}
  \frac{1}{\sqrt{2\pi}}
  \int_{-\infty}^{\infty} \dif x \, e^{-x^2/2} \tanh^2 \del{\frac{x}{T_c}}
  = \frac{1}{z}.
  \label{eq:viana-bray-Tc}
\end{equation}
For our chosen value $z=6$, we find
% TODO: why 6?
\begin{equation}
  T_c(z=6) \approx 2.0564 \quad\text{(diluted)}.
  \label{eq:viana-bray-Tc-z6}
\end{equation}

\section{Method}
\label{sec:nonextensive-method}

We perform Monte Carlo simulations of the models described in
\cref{sec:nonextensive-models}. To speed up equilibration, we use the
parallel-tempering Monte Carlo method described in
\cref{sec:numerical-parallel-tempering}.

To ensure that measurements are performed in equilibrium we use the
equilibration test for Gaussian spin glasses described in
\cref{sec:numerical-equilibration}. That is, we successively double the number
of Monte Carlo sweeps, each time averaging over the last half of the sweeps,
until (at least) the last three data points for
$\Delta(\widetilde{U},\widetilde{q_l})$ are consistent with zero. The total
number of sweeps used in this check is shown as $N_{\mathrm{equil}}$ in
\cref{tab:nonextensive-params-c,tab:nonextensive-params-d}. We perform the
equilibration test only for the largest sizes (which account for most of the
overall simulation time) and use the same value of $N_{\mathrm{equil}}$ for the
smaller sizes.

We then do measurement runs where, after an initial $N_{\mathrm{equil}}$ sweeps
to ensure the system has reached equilibrium, we do an additional 10 to 20
times as many sweeps during which measurements are performed. The detailed
parameters of the simulations are given in
\cref{tab:nonextensive-params-c,tab:nonextensive-params-d}. To avoid bias in
measurements with distinct thermal averages, e.g. \cref{eq:binder-ratio}, each
thermal average is evaluated on a separate replica of the system with the same
realization of the random couplings. Since the quantities of interest have no
more than two distinct thermal averages, we simulate two copies of the system
at each temperature.
%TODO: distinct thermal averages in Numerical Methods?

\pgfplotstableset{
  nonextensive params/.style={
    /pgf/number format/set thousands separator={},
    columns/sigma/.style={
      column name={$\sigma$},
      dec sep align,
      precision=3
    },
    columns/L/.style={
      column type={r},
      column name={$L$}},
    columns/nsamp/.style={
      column type={r},
      column name={$N_{\mathrm{samp}}$}
    },
    columns/neq/.style={column name={$N_{\mathrm{equil}}$}},
    columns/nmeas/.style={column name={$N_{\mathrm{meas}}$}},
    columns/Tmin/.style={
      column name={$T_{\mathrm{min}}$},
      dec sep align,
      precision=3
    },
    columns/Tmax/.style={
      column name={$T_{\mathrm{max}}$},
      dec sep align,
      precision=3
    },
    columns/nT/.style={column name={$N_T$}},
  }}

\begin{table}
  \centering
  \pgfplotstabletypeset[nonextensive params]{data/params-c.csv}
  \caption[%
    Simulation parameters for one-dimensional long-range models with undiluted
    bonds.
  ]
  {%
    Simulation parameters for the undiluted models. $N_{\mathrm{samp}}$ is the
    number of samples; $N_{\mathrm{equil}}$ and $N_{\mathrm{meas}}$ are the
    numbers of sweeps used for the equilibration and measurement phases
    respectively. We simulate $N_T$ logarithmically-spaced temperatures between
    $T_{\mathrm{min}}$ and $T_{\mathrm{max}}$.
  }
  \label{tab:nonextensive-params-c}
\end{table}

\begin{table}
  \centering
  \pgfplotstabletypeset[nonextensive params]{data/params-d.csv}
  \caption[%
    Simulation parameters for one-dimensional long-range models with diluted
    bonds.
  ]
  {Simulation parameters for the diluted models.}
  \label{tab:nonextensive-params-d}
\end{table}
%TODO: define style for data tables in one place
%TODO: table formatting


We consider moments of the spin glass order parameter,
\begin{equation}
  q = \frac{1}{L} \sum_i S_i^{(1)} S_i^{(2)},
\end{equation}
where ``(1)" and ``(2)" refer to independent replicas of the system with the
same realization of the random couplings. Of particular interest are the
spin-glass susceptibility
\begin{equation}
  \sgsusc = L\av{q^2}
  \label{eq:sgsusc}
\end{equation}
and the Binder ratio,
\begin{equation}
  g = \frac{1}{2}\del{3 - \frac{\av{q^4}}{\av{q^2}^2}},
  \label{eq:binder-ratio}
\end{equation}
where $\av{\cdots}$ indicates a thermal average and an average over the disorder.

Because the Binder ratio is dimensionless, its finite-size scaling behavior is
simple (see \cref{sec:numerical-fss}). The models considered here are all in
the mean-field regime, in which the scaling relation takes the form
\begin{equation}
  g \sim \widetilde{g}\sbr{L^{1/3} \del{T-T_c}}
  \label{eq:binder-scaling}
\end{equation}
(where $\sim$ indicates asymptotic equivalence for large $L$).
Although the spin-glass susceptibility is not dimensionless, its scaling form
is also known exactly in the mean-field regime,%
\footnote{%
  For a discussion of how standard finite-size scaling is modified in the
  mean-field regime, see, for example,
  \autocite{binder1985finite, luijten1999finite, jones2005finite,
    brezin1982investigation, brezin1985finite}
}
\begin{equation}
  \sgsusc \sim L^{1/3} \widetilde{\chi}\sbr{L^{1/3} \del{T-T_c}}.
  \label{eq:sgsusc-scaling}
\end{equation}
% TODO: discuss this a bit more

Thus, at least asymptotically for large $L$, \emph{the Binder ratio $g$ and
  scaled susceptibility $\sgsusc/L^{1/3}$ are independent of $L$ at $T=T_c$}.
We can exploit this to estimate $T_c$ from the intersections of the data for
different sizes when plotted against $T$. However, we will find that the data
for different pairs of sizes do not intersect at a common temperature (see, for
example, \cref{fig:binder-c}), but rather that data for each pair $(L_1,L_2)$
intersect at a size-dependent temperature $T^*(L_1,L_2)$. This is because the
corrections to the asymptotic scaling relations become significant for the
range of sizes we are able to simulate.

According to standard finite-size scaling the spin-glass susceptibility
scales near the critical point as
\begin{equation}
    \sgsusc(t,L)
    = L^a \sbr{f(L^b t) + L^{-\omega} g(L^y t) + \cdots}
      + c_0 + c_1 t + \cdots,
    \label{eq:standard-fss}
\end{equation}
\autocite{privman1983finite},
where $t=T-T_c$, $a=2-\eta$ ($=2\sigma-1$ here), and $b=1/\nu$. The
$L^{-\omega}$ term is the leading \emph{singular} correction to scaling and
$c_0$ is the leading \emph{analytic} correction to scaling.

In the mean-field regime, $\sigma<\sigma_u=2/3$, the exponents $a$ and $b$ are
independent of $\sigma$ and take their values at $\sigma_u$ for all
$1/2<\sigma<\sigma_u$.%
\footnote{%
  This is explained by the presence of a ``dangerous irrelevant variable," see
  \textcite{binder1985finite,luijten1999finite,jones2005finite} and also
  the discussion in \cref{sec:fss-intro}.
}
Although the $L^{2\sigma-1}$ term is therefore replaced as the \emph{largest}
term by an $L^{1/3}$ term, we expect the former not to disappear but instead
become a correction to scaling. Therefore, for $\sigma<\sigma_u$ we replace
\cref{eq:standard-fss} with
\begin{align}
  \sgsusc(t,L)
  &= L^{1/3}\sbr{f(L^{1/3}t) + L^{-\omega} g(L^{1/3}t) + \cdots} \\
  &+ d_0 L^{2\sigma-1} h(L^{1/3}t) + c_0 + c_1 t + \cdots.
\end{align}
% TODO: understand "h g" in above eqn
The correction exponent $\omega$ can be obtained in the mean-field regime from
the work of \textcite{kotliar1983one} and is given by $\omega=2-3\sigma$. Thus,
in the nonextensive regime $\sigma<1/2$, the dominant correction to scaling is
the constant $c_0$. Including the dominant correction, \cref{eq:sgsusc-scaling}
becomes
\begin{equation}
  \sgsusc \sim L^{1/3} \widetilde{\chi}\sbr{L^{1/3}\del{T-T_c}} + c_0.
  \label{eq:sgsusc-scaling-corrected}
\end{equation}
Data for $\sgsusc/L^{1/3}$ for a pair of sizes $(L,2L)$ will intersect at a
temperature $T^*(L,2L)$ where
\begin{equation}
  \widetilde{\chi}\sbr{L^{1/3}(T^*-T_c)} + c_0/L^{1/3}
  =\widetilde{\chi}\sbr{(2L)^{1/3}(T^*-T_c)} + c_0/(2L)^{1/3}.
\end{equation}
Expanding $\widetilde{\chi}$ to first order and solving for $T^*$, we obtain
the correction to the intersection temperatures corresponding to the dominant
correction to scaling,
\begin{equation}
  T^*(L,2L) = T_c + A/L^{2/3} + \cdots,
  \label{eq:Tx-corrected}
\end{equation}
where $A$ is a constant. We expect the corresponding result for the Binder
ratio $g$ to have the same form.
% TODO: why?
In the following analysis we will fit \cref{eq:Tx-corrected} to the data to
estimate the bulk transition temperature $T_c$.


\section{Results}

We first present our results for the undiluted model. Data for the Binder ratio
and the scaled spin glass susceptibility are shown for $\sigma=0$ (SK model)
and $\sigma=0.25$ in \cref{fig:binder-c,fig:chi-c}. Note the large corrections
to scaling for the Binder ratio and the relatively small corrections for the
scaled susceptibility.

\begin{figure}
  \centering
  \begin{subfigure}{0.49\textwidth}
    \centering
    \includestandalone{figures/binder-c-s00}
    \subcaption{$\sigma=0$}
  \end{subfigure}
  \begin{subfigure}{0.49\textwidth}
    \centering
    \includestandalone{figures/binder-c-s25}
    \subcaption{$\sigma=0.25$}
  \end{subfigure}
  \caption[
    Data for the Binder ratio for the one-dimensional undiluted spin glass.
  ]
  {
    Data for the Binder ratio for the undiluted model. The exact value of the
    transition temperature for the SK model, $T_c=1$, is marked with a vertical
    line.
  }
  \label{fig:binder-c}
\end{figure}

\begin{figure}
  \centering
  \begin{subfigure}{0.49\textwidth}
    \centering
    \includestandalone{figures/chi-scaled-c-s00}
    \subcaption{$\sigma=0$}
  \end{subfigure}
  \begin{subfigure}{0.49\textwidth}
    \centering
    \includestandalone{figures/chi-scaled-c-s25}
    \subcaption{$\sigma=0.25$}
  \end{subfigure}
  \caption[
    Data for the Binder ratio and the scaled spin-glass susceptibility for the
    one-dimensional undiluted spin glass with $\sigma=0$ and $\sigma=0.25$.
  ]
  {
    Data for the scaled spin-glass susceptibility for the undiluted model. The
    exact value of the transition temperature for the SK model, $T_c=1$, is
    marked with a vertical line.
  }
  \label{fig:chi-c}
\end{figure}

\Cref{fig:Tx-c} shows the intersection temperatures for the undiluted models,
determined by first fitting a cubic spline to the data and finding the
intersections of the splines. The error bars were estimated using the bootstrap
resampling method (discussed in \cref{sec:numerical-bootstrap}). We then fit
\cref{eq:Tx-corrected} to the intersection temperatures to estimate $T_c$. For
\emph{both} values of $\sigma$ we obtain a value consistent with $T_c=1$ (with
very small errors), the exact value for the SK model. The quality of the fit,
as quantified by the goodness of fit parameter $Q$
\autocite{press2007numerical}, is satisfactory except for the Binder ratio data
for the SK model. We do not have a good explanation of this, except perhaps
that multiple corrections to scaling are significant for the range of sizes
studied. In any case we note that the result $T_c=1$ for the SK model is
rigorously correct. The result that $T_c=1$ also for the model with
$\sigma=0.25$, at the midpoint of the nonextensive region, provides strong
evidence for the claim of \textcite{mori2011instability} that all models in the
nonextensive region are identical to the SK model.
% PETER: While it would be useful to check this also in the space glass phase
% below $T_c$, such simulations would be difficult because relaxation times
% increase dramatically at low $T$, and so the range of sizes that could be
% studied would be much more limited than in the data presented here.
% TODO: space glass?
% TODO: bootstrap vs. jackknife

\begin{figure}
  \centering
  \begin{subfigure}{0.49\textwidth}
    \centering
    \includestandalone{figures/Tx-c-s00}
    \subcaption{$\sigma=0$}
  \end{subfigure}
  \begin{subfigure}{0.49\textwidth}
    \centering
    \includestandalone{figures/Tx-c-s25}
    \subcaption{$\sigma=0.25$}
  \end{subfigure}
  \caption[
    Results for the intersection temperatures for the one-dimensional undilted
    model with $\sigma=0$ (SK model) and $\sigma=0.25$.
  ]
  {
    Results for the intersection temperatures for the undiluted model.
  }
  \label{fig:Tx-c}
\end{figure}

The corresponding results for the diluted models with $\sigma=0$ and $0.25$ are
shown in \cref{fig:binder-d,fig:chi-d,fig:Tx-d}. For the diluted model we also
performed simulations with $\sigma=0.375$ and show the resulting intersection
temperatures in \cref{fig:Tx-d-375}. For $\sigma=0$, the Viana-Bray (VB) model,
the transition temperature is given by \cref{eq:viana-bray-Tc}, which, for
$z=6$ taken here, gives the result in \cref{eq:viana-bray-Tc-z6}. Here again we
see that the corrections to scaling are larger for the Binder ratio than for
the scaled spin glass susceptibility. Fitting \cref{eq:Tx-corrected}, we find a
predicted value of $T_c$ consistent with the exact value for the VB model for
\emph{all} values of $\sigma$.

\begin{figure}
  \centering
  \begin{subfigure}{0.49\textwidth}
    \centering
    \includestandalone{figures/binder-d-s00}
    \subcaption{$\sigma=0$}
  \end{subfigure}
  \begin{subfigure}{0.49\textwidth}
    \centering
    \includestandalone{figures/binder-d-s25}
    \subcaption{$\sigma=0.25$}
  \end{subfigure}
  \caption[
    Data for the Binder ratio for the one-dimensional undiluted spin glass.
  ]
  {
    Data for the Binder ratio for the diluted model. The Viana-Bray model
    transition temperature $T_c \approx 2.056$, obtained from
    \cref{eq:viana-bray-Tc}, is marked with a vertical line.
  }
  \label{fig:binder-d}
\end{figure}

\begin{figure}
  \centering
  \begin{subfigure}{0.49\textwidth}
    \centering
    \includestandalone{figures/chi-scaled-d-s00}
    \subcaption{$\sigma=0$}
  \end{subfigure}
  \begin{subfigure}{0.49\textwidth}
    \centering
    \includestandalone{figures/chi-scaled-d-s25}
    \subcaption{$\sigma=0.25$}
  \end{subfigure}
  \caption[
    Data for the Binder ratio and the scaled spin-glass susceptibility for the
    one-dimensional diluted spin glass with $\sigma=0$ and $\sigma=0.25$.
  ]
  {
    Data for the scaled spin-glass susceptibility for the diluted model. The
    Viana-Bray model transition temperature $T_c \approx 2.056$, obtained from
    \cref{eq:viana-bray-Tc}, is marked with a vertical line.
  }
  \label{fig:chi-d}
\end{figure}

\begin{figure}
  \centering
  \begin{subfigure}{0.49\textwidth}
    \centering
    \includestandalone{figures/Tx-d-s00}
    \subcaption{$\sigma=0$}
  \end{subfigure}
  \begin{subfigure}{0.49\textwidth}
    \centering
    \includestandalone{figures/Tx-d-s25}
    \subcaption{$\sigma=0.25$}
  \end{subfigure}
  \caption[
    Results for the intersection temperatures for the one-dimensional diluted
    model with $\sigma=0$ (VB model) and $\sigma=0.25$.
  ]
  {
    Results for the intersection temperatures for the diluted model.
  }
  \label{fig:Tx-d}
\end{figure}

\begin{figure}
  \centering
  \includestandalone{figures/Tx-d-s38}
  \caption[
    Results for the intersection temperatures for the diluted model with
    $\sigma=0.375$.
  ]
  {
    Results for the intersection temperatures for the diluted model with
    $\sigma=0.375$.
  }
  \label{fig:Tx-d-375}
\end{figure}


\section{Conclusions}

We have performed Monte Carlo simulations to investigate the transition
temperatures of one-dimensional Ising spin glasses, both undiluted and diluted,
for several values of $\sigma$ in the nonextensive regime $0 \leq \sigma <
1/2$. For the undiluted model we studied two values of $\sigma$, $\sigma=0$ and
$\sigma=0.25$. For $\sigma=0.25$, which lies in the middle of the nonextensive
region, we find that the transition temperature agrees to high precision with
the exact solution of the SK model. As a check, we also simulated the
$\sigma=0$ case, obtaining results consistent with the exact SK model result,
though there seem to be multiple corrections to FSS for some of the data.

For the diluted model we studied three values of $\sigma$: $\sigma=0$, which
corresponds to the Viana-Bray model; $\sigma=0.25$, which lies in the middle of
the nonextensive region, and $\sigma=0.375$. In all cases we found the
transition temperature to be consistent with the exact solution of the
Viana-Bray model; all results were within $\sim 1.5$ standard deviations.

To conclude, our results provide confirmation of the proposal
\autocite{mori2011instability} that the behavior of (undiluted) spin glasses
everywhere in the nonextensive regime is identical to that of the SK model. We
have also proposed and provided evidence that an analogous result applies to
\emph{diluted} spin glass models.
