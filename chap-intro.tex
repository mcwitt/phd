\chapter{Introduction}

\setlength\epigraphwidth{0.4\textwidth}
{
  \SingleSpacing
  \epigraph{
    The behavior of large and complex aggregates of elementary particles, it
    turns out, is not to be understood in terms of a simple extrapolation of the
    properties of a few particles.
%   Instead, at each level of complexity entirely new properties appear, and the
%   understanding of the new behaviors requires research which I think is as
%   fundamental in its nature as any other.
  }{Philip W. Anderson \\ \textit{More is Different} (1972)}
}


\section{Motivation}

With the widespread adoption and rapid advancement of the digital computer in
the mid-20th century have come new, powerful approaches to modeling systems
using numerical simulations. Before the advent of computers, models of physical
systems were restricted to those that could be solved analytically, or at least
those for which a suitable approximation, or expansion in some small parameter,
can be made while retaining the essential physics. Numerical simulations expand
the space of practical models significantly, while simultaneously bringing a
different set of challenges and limitations. With the present technology
numerical simulations of \emph{classical} systems are limited to sizes many
orders of magnitude below the macroscopic scale. For \emph{quantum} systems the
situation is much more difficult, as the computational resources required
expand exponentially in the size of the system; furthermore, simulations of
fermionic systems suffer from the ``sign problem" in evaluating oscillatory
integrals which also induces an exponential penalty in the size of the system.
Nevertheless, numerical simulations have become a crucial complement to
traditional theory and experiment as computing technology has advanced,
allowing the simulation of progressively larger systems, and techniques such as
finite-size scaling analysis (described in \cref{sec:numerical-fss}) have been
developed to extrapolate results for finite systems into the thermodynamic
(infinite-size) limit. It is hoped that an eventual practical \emph{quantum
  computer} could efficiently simulate quantum systems.

Numerical simulations have been especially useful in the study of ``complex
systems," systems consisting of a large number of \emph{strongly-interacting}
parts whose aggregate behavior is strongly nonlinear, meaning (roughly) that
the overall behavior cannot be approximated simply as the sum of its parts.
Examples of such systems in science are numerous, including models of
high-temperature superconductivity in many-body physics, the behavior of ant
colonies and the interaction of cells in the immune system in biology, and
neural networks in neuroscience and machine learning. In this thesis we focus
on models of magnetism in statistical physics, which are among the simplest
examples of complex systems to study from an analytical and numerical
perspective.


\section{Ising model of ferromagnetism}
\label{sec:intro-ising}

In the early 1920s, William Lenz gave his student Ernst Ising a simple
mathematical model of ferromagnetism, now known as the \emph{Ising model},
which was to be the topic of Ising's dissertation. The model consists of
discrete variables, ``spins," which can take one of two values, $+1$ or $-1$,
and are arranged on a regular lattice with pairwise nearest-neighbor
interactions. In addition to the nearest-neighbor interactions, the spins also
interact with an externally-applied field $H_i$, which in general can be
nonuniform. The Hamiltonian is given by
\begin{equation}
  \ham = -\frac{1}{2} \sum_{ij} J_{ij} S_i S_j - \sum_i H_i S_i,
  \label{eq:ising-ham}
\end{equation}
where the first sum is over all pairs of spins, and $J_{ij}=J$, a positive
constant, if $i$ and $j$ are nearest neighbors on the lattice and 0 otherwise.

As the temperature is decreased through the Curie temperature $T_c$, a
ferromagnet in zero external field undergoes a continuous phase transition from
the \emph{paramagnetic} phase, in which the magnetization is zero, to the
\emph{ferromagnetic} phase, in which the magnetization is finite and can be
either positive or negative. This is an example of spontaneous symmetry
breaking, where a symmetry, in this case time-reversal symmetry $S_i \to -S_i$,
is ``broken" as the system undergoes a transition into a phase with lower
symmetry. In his 1924 dissertation, Ising solved the one-dimensional version
of the model that now bears his name and showed that it did \emph{not} have a
ferromagnetic phase at any positive temperature. From this he (incorrectly)
concluded that the model does not admit a finite-temperature phase transition
in any number of dimensions, and thus was not a suitable model of
ferromagnetism \autocite{ising1925beitrag}.

This assumption was proven incorrect by \textcite{peierls1936ising}, who gave
an argument based on the free energy of \emph{domain walls} (the boundaries
separating regions of positive and negative magnetization) showing that phase
transitions \emph{do} occur in the model for dimensions greater than two. The
argument goes as follow. In one dimension, the energy cost of a domain wall is
$2J$ since the energy of a pair of spins has magnitude $J$ and changes sign
when one of them is flipped. But the entropy added by the domain wall goes like
the logarithm of the number of spins $L$, since there are $L$ places%
\footnote{assuming periodic boundary conditions} where it could go. Thus, in
one dimension, the \emph{free energy cost} of a domain wall is negative for any
positive temperature in the infinite-size limit, so the system will form domain
walls on all scales, destroying the ferromagnetic order. However, in two
dimensions, the energy cost of a domain wall is $2 J n$, where $n$ is the
number of spins on the boundary. An upper bound on the number of ways to form a
closed loop is $(z-1)^n$, where $z$ is the coordination number of the lattice.
This is an upper bound because it allows for self-intersections, which are not
possible for a real domain wall. The free-energy cost $\Delta F_n$ to form a
domain wall of length $n$ is then bounded from below as
\begin{equation}
  \Delta F_n > \sbr{2J - T \log(z-1)} n.
\end{equation}
Thus, for small positive $T$ the formation of domain walls of any size is
suppressed, and the system will develop ferromagnetic order. This also gives a
lower bound on the transition temperature $T_c$ at which the system transitions
to ferromagnetic order, $T_c > 2J/\log(z-1) \approx 1.82 J$ in two dimensions.
A similar argument can be made for all dimensions $d \geq 2$.

It turns out that it is a general feature of phase transitions in diverse
systems that there is a number of dimensions $d_l$, called the \emph{lower
  critical dimension}, below which a transition does not occur at finite
temperature, so the above result can be stated as $d_l=2$ for the Ising model.
In fact, this result holds for all systems in the Ising universality class,
which includes systems with a scalar order parameter near a continuous phase
transition, including (\textit{e.g.}) magnetic systems and liquids. We will
motivate the phenomenon of universality in the discussion of Landau theory in
\cref{sec:intro-landau}; it is now understood within the framework of the
\emph{renormalization group}.

Two decades after the introduction of the Ising model, in a remarkable
mathematical achievement, \textcite{onsager1944crystal} gave an analytic
solution of the two-dimensional Ising model in zero field. The exact
transition temperature is given by%
\begin{equation}
  T_c = \frac{2J}{\log(1 + \sqrt{2})} \approx 2.269 J.
  \label{eq:ising-Tc-exact}
\end{equation}
% TODO: exact T_c was found earlier
Onsager's solution is very complicated and the techniques do not generalize
well. Subsequent attempts at an analytical solution in two dimensions with a
field or in three and higher dimensions have been unsuccessful, and in fact
recently arguments have been made from computational complexity theory that
these problems are intractable.%
\footnote{%
  \textcite{istrail2000statistical} shows that computing the partition function
  for $d=2$ with a field or for $d>2$ belongs to the class of NP-complete
  problems, which are unlikely to have closed-form solutions.
}

The Ising model is ubiquitous in the literature of statistical mechanics
because it is a relatively uncomplicated model which, thanks to the phenomemon
of \emph{universality}, exhibits critical behavior \emph{identical} to other
systems in the same number of dimensions and with \emph{scalar} order
parameters, and thus provides a convenient setting for the study of phase
transitions in a wide variety of systems.


\subsection{Mean-field theory}

Despite the apparent simplicity of Ising model, exact solutions do not appear
to be possible in dimensions $d>2$. The source of the difficulty is the
spin-spin interactions, which preclude a solution by factorizing the partition
function as a product of single-site factors. This is a general problem in
many-body statistical physics; as a result, exact analytical solutions are few
and far between.

\emph{Mean-field theory} (MFT) is the simplest possible approximation of an
interacting system aside from ignoring the interactions altogether. The
resulting model is usually straightforward to solve analytically, although
there are exceptions (for example, the Sherrington-Kirkpatrick model discussed
in \cref{sec:intro-sk}.) There are many equivalent ways to generate a
mean-field theory for a given system, but all are based on the same essential
approximation of ignoring the interaction of \emph{fluctuations}.

For example, following \textcite{cardy1996scaling}, we can generate a mean-field
theory of the Ising ferromagnet as follows. We write for the interacting part
of the Hamiltonian
\begin{align*}
  \ham_{\mathrm{int}}
  &\equiv -\frac{1}{2} \sum_{ij} J_{ij} S_i S_j \\
  &= -\frac{1}{2} \sum_{ij} J_{ij} \sbr{M + (S_i - M)}\sbr{M + (S_j - M)} \\
  &\approx -\frac{1}{2} \sum_{ij} J_{ij} \sbr{-M^2 + M(S_i + S_j)} \\
  &= \frac{1}{2} N J_z M^2 - J_z M \sum_i S_i,
\end{align*}
where $M \equiv \av{S}$ is the magnetization and $J_z=\sum_j J_{ij}$. The
mean-field approximation comes in third step, where we have discarded the term
which is quadratic in the fluctuations, \textit{i.e.} $(S_i - M)(S_j - M)$.
Thus we have decoupled the spin degrees of freedom and can solve the problem by
factorizing the partition function,
\begin{align*}
  Z
  &= \Tr\exp\sbr{-\frac{1}{2} \beta N J_z M^2 - \beta(H + J_z M) \sum_i S_i} \\
  &= e^{-\frac{1}{2} \beta N J_z M^2} \sbr{2 \cosh \beta(H + J_z M)}^N.
\end{align*}
The free energy per spin is then
\begin{equation}
  f = \frac{1}{2} J_z M^2 - \beta^{-1}\log\cosh\beta(H + J_z M),
  \label{eq:mf-fe}
\end{equation}
dropping the constant which vanishes in the thermodynamic limit. This
expression depends on the magnetization $M$, which we have not yet specified.
But since $M = -\partial f / \partial H$, we have the following
\emph{self-consistency condition}
\begin{equation}
  M = \tanh\beta(H + J_z M).
  \label{eq:mf-scc}
\end{equation}
Note that this has the same form as the Curie law, which we would expect to
obtain had we set the spin-spin couplings $J_{ij}=0$, but with the external field
$H$ replaced by $H + J_z M$. The additional term $J_z M$ has the simple
interpretation as the ``molecular field" generated by a spin's neighbors,
neglecting fluctuations about the the mean $M$. Note that for $H=0$ and high
temperature (small $\beta$), \cref{eq:mf-scc} has only a single solution at
$M=0$, corresponding to the paramagnetic state, while at low temperatures two
additional solutions appear at $M=\pm M_0$ corresponding to the two pure states
of the paramagnetic phase. Thus the mean field model demonstrates a continuous
phase transition with spontaneous symmetry breaking. The (mean-field)
transition temperature $\tcmf$ is given by the limit $M \to 0^+$ in
\cref{eq:mf-scc}, from which we obtain $M = J_z M / \tcmf$, or
\begin{equation}
  \tcmf = \sum_j J_{ij}.
\end{equation}
For the Ising model on a hypercubic lattice $J_z \equiv \sum_j J_{ij} = 2 d J$
and the corresponding result for the two-dimensional Ising model is $\tcmf=4J$,
an overestimate of the exact transition temperature [see
\cref{eq:ising-Tc-exact}]. Mean-field theories give an upper bound on the
transition temperature because they neglect the effect of fluctuations, which
can drive the transition to lower temperatures. Hence, from the Peierls
argument described above and mean-field theory, we have (correctly) bounded the
transition temperature for the two-dimensional Ising model as $1.82 \lessapprox
T_c/J < 4$.

To understand the nature of the broken symmetry it is helpful to consider the
behavior of the mean-field free energy near the transition. In zero field,
\cref{eq:mf-fe} has the expansion
\begin{equation}
  f = \frac{1}{2} J_z (1 - \beta J_z) M^2 + \mathcal{O}(M^4),
  \label{eq:mf-fe-trunc}
\end{equation}
where the coefficient of the $M^4$ term is positive. As shown in
\cref{fig:mf-fe}, for temperatures above $T_c$ the coefficient of the
quadratic term is positive and there is a single minimum at $M=0$,
corresponding to the paramagnetic state. As the temperature is lowered through
$T_c$, the $M=0$ solution of the minimum equation becomes unstable and is
replaced by two new minima at $\pm M_0(t)$ which shift continuously from $M=0$
as the temperature is lowered. The system will choose one of these minima,
which correspond to the ``up" and ``down" pure states of the ferromagnetic
phase. This situation describes a continuous phase transition with spontaneous
symmetry breaking. Now consider what happens when a nonzero external field $H$
is applied. For $T<T_c$, as $H$ passes through zero from negative to positive,
the global minimum of the free energy shifts discontinuosly from $-M_0$ to
$M_0$. This exemplifies a first-order phase transition.

Phase transitions are characterized by a set of \emph{critical exponents},
which describe how various quantities vary asymptotically near the critical
point. For example, in magnetic systems, the critical exponent $\beta$
is defined as
\begin{equation}
  M \propto (T_c - T)^{\beta}
\end{equation}
in the limit $T \to T_c^-$.
We can estimate this exponent from the mean-field theory developed above.
Expanding \cref{eq:mf-fe} in $M$, we find the form
\begin{equation}
  f \propto a t M^2 + \frac{1}{2} b M^4 - H M,
  \label{eq:mf-fe-form}
\end{equation}
where $a$, $b$, and $c$ are constants which depend only weakly on the reduced
temperature $t=(T-T_c)/T_c$ \autocite{cardy1996scaling}. Minimizing with
respect to $M$ we find that, near $T_c$, $M^2=(a/2b)t$, from which we read off
the mean-field value of the critical exponent $\beta=1/2$. This is not all that
accurate compared with values measured in experiment, $\beta \approx 0.32$, but
this is not surprising given the seemingly crude approximation that led to this
result.

What \emph{is} surprising is that the determination of the critical exponents
from mean-field theory seems not to rely on any details of the model,
\textit{e.g.} the coefficients in \cref{eq:mf-fe-form}, but only on the form of
the expansion. In fact, Landau showed that such a form follows for systems with
a scalar order parameter undergoing a continuous phase transition, from only
simple symmetry arguments and assumptions of analyticity. Thus the disagreement
between the critical exponents predicted by Landau theory and those observed in
experiment was at first considered paradoxical, seemingly a violation of
dimensional analysis \autocite{goldenfeld1992lectures}. The existence of
so-called ``anomalous dimensions," the differences between critical exponents
observed in experiment and those predicted by Landau theory, was to be
explained by the renormalization group. In the next section we will give a
brief review of Landau theory and discuss how it breaks down near a critical
point.

Finally, we emphasize that there are many ways to generate equivalent
mean-field theories for a given system. For example, we could have arrived at
\cref{eq:mf-scc} by supposing that each spin behaves as an isolated spin in an
external field equal to $H + J_z M$, where, ignoring fluctuations, we
approximate the contribution of the neighbors as $J_z M$. Note that this is
\emph{equivalent} to an (infinite-size) model where \emph{every} pair of spins
is coupled by an interaction $J/N$, because the thermal average $M$ is
equivalent to the average over an infinite number of neighbors in the latter
case.

\begin{figure}
  \centering
  \begin{subfigure}{0.49\textwidth}
    \centering
    \begin{tikzpicture}
      \begin{axis}[
        schematic,
        width=0.7\figurewidth,
        domain=-1.2:1.2,
        ymin=-1.2,ymax=1.2,
        xlabel={$M$},ylabel={$y$},
      ]
        \addplot+[black,dashed,name path=M] {x}
          node[above] {$y=M$};
        \addplot+[thick,red] {tanh(4*x/7)}
          node[right] {$T>T_c$};
        \addplot+[thick,blue,name path=RHS] {tanh(7*x/4)}
          node[right] {$T<T_c$};
        \path[name intersections={of=M and RHS}];
        \fill
          (intersection-1) circle (2pt)
          (intersection-3) circle (2pt);
        \draw[thick] (intersection-2) circle (2pt);
      \end{axis}
    \end{tikzpicture}
    \subcaption{}\label{fig:mf-scc}
  \end{subfigure}
  \begin{subfigure}{0.3\textwidth}
    \centering
    \begin{tikzpicture}
      \begin{axis}[
        schematic,
        width=0.7\figurewidth,
        domain=-2.5:2.5,
        xlabel={$M$},ylabel={$f$}
      ]
        \addplot+[brown] {x^4}
          node[above right] {$T=T_c$};
        \addplot+[blue] {-4*x^2 + x^4}
          node[above right] {$T<T_c$};
        \addplot+[red] {+4*x^2 + x^4}
          node[above right] {$T>T_c$};
        \draw[thick] (0,0) circle (2pt);
        \fill
          ({-2^(1/2)},-4) circle (2pt)
          ({+2^(1/2)},-4) circle (2pt);
      \end{axis}
    \end{tikzpicture}
    \subcaption{}\label{fig:mf-fe}
  \end{subfigure}
  \caption[
    Plots of the mean-field self-consistency condition and the mean-field free
    energy for the Ising model.
  ]
  {
    Panel \subref{fig:mf-scc} shows the right-hand side of the self-consistency
    condition, \cref{eq:mf-scc}, as a function of the magnetization $M$ for $H
    =0$. Above the critical temperature $T_c$ there is a single self-consistent
    solution corresponding to zero magnetization, while below $T_c$ there are
    two solutions with nonvanishing magnetization $M=\pm M_0$ in addition to
    the solution at $M=0$. Panel \subref{fig:mf-fe} shows the free energy as a
    function of $M$, showing that for $T<T_c$, the solution with $M=0$ is
    \emph{unstable}.
  }
\end{figure}


\subsection{Landau theory}
\label{sec:intro-landau}

The Landau theory of phase transitions is a phenomenological theory based on
a few simple assumptions about the behavior of the relevant order parameter
$\eta$ in the vicinity of a phase transition. We define a function $L(\eta)$,
called the Landau free energy, whose global minimum is supposed to give the
equilibrium state of the system, and postulate that
\begin{enumerate}
  \item The order parameter $\eta=0$ in the disordered phase; $\eta$ is small
    and nonzero in the ordered phase close to the transition.
    \label{itm:eta}
  \item $L$ respects the symmetries of the underlying Hamiltonian.
  \item $L$ is an analytic function of both $\eta$ and the set of couplings
    $\cbr{K}$ describing the interactions; thus it can be expanded as a power
    series in $\eta$ near the critical point.
  \item $L$ is a local function of $\eta$, \textit{i.e.} it only involves
    derivatives to a finite order.
\end{enumerate}

For the Ising model an appropriate choice of order parameter is $\eta=M$, the
magnetization. Because of the time-reversal ($S_i \to -S_i$) symmetry of the
Hamiltonian, we require that the Landau free energy $\mathcal{L}(\eta)$ be
symmetric about $\eta=0$ in the absence of an external field. Expanding
$\mathcal{L}(\eta)$ in powers of $\eta$ near the critical point, we obtain the
form
\begin{equation}
  L = a t \eta^2 + \frac{1}{2} b \eta^4 - H \eta,
  \label{eq:landau-fe}
\end{equation}
where the last term is the leading-order contribution to the energy in a field
$H$ and, as can be seen from \cref{itm:eta}, $a$ and $b$ are independent of the
reduced temperature $t$ to leading order. Note that this is the same form as
\cref{eq:mf-fe-form}, which we derived before by solving a mean-field model of
Ising model Hamiltonian. Here, we arrive at the same form using only the
postulates of Landau theory. The only details of the Ising model that were
relevant to the derivation are the fact that the order parameter is a scalar,
and that the Hamiltonian obeys time-reversal symmetry in zero field.

To calculate correlation functions in Landau theory, we allow $\eta$ to vary in
space. We may define a spatially-varying order parameter for the Ising model by
using a coarse-graining procedure whereby $\eta(\vec{r})$ is defined to be the
average of the spins in a block of size $\Lambda^{-1}$ centered at $\vec{r}$.
Choosing $\Lambda^{-1} \sim \xi$, the correlation length, ensures that spins
within a given block have similar values. We can then postulate that the
contribution to the Landau free energy due to the interaction of adjacent
blocks has the simplest-possible \emph{analytic} form, $\sbr{\eta(\vec{r}) -
  \eta(\vec{r} + \vec{\delta})}^2/\Lambda^{-2}$, where $\vec{\delta}$ is the
vector from the center of one block to the other, and
$\abs{\vec{\delta}}=\Lambda^{-1}$. Taking the continuum limit, \textit{i.e.}
the lattice spacing $a \to 0$, and noting that $\eta(\vec{r})$ is
slowly-varying in space, we define coarse-grained \emph{Ginzburg-Landau free
  energy},
\begin{equation}
  L_{\Lambda}[\eta]
  = \int\dif^d \vec{r}
  \cbr{\frac{\gamma}{2}\del{
      \nabla\eta}^2 + a t \eta^2 + \frac{1}{2} b \eta^4 - H \eta},
  \label{eq:landau-cg}
\end{equation}
where $\gamma$ is a constant describing the strength of the interaction between
neighboring blocks. Note that $\eta(\vec{r})$ and $L_{\Lambda}[\eta(\vec{r})]$
both depend on the length scale $\Lambda^{-1}$ over which the coarse graining
is performed. To make a connection between the Ginzburg-Landau free energy and
the \emph{thermodynamic} free energy, we note that
\begin{equation}
  e^{-\beta L_{\Lambda}[\eta(\vec{r})]} = \Tr^{\prime} e^{-\beta H \cbr{S_i}},
\end{equation}
where $\Tr^{\prime}$ indicates that the trace is to be taken over only those
microscopic configurations which are compatible with the coarse-grained order
parameter $\eta(\vec{r})$. Then the partition function is obtained by averaging
over all configurations of the coarse-grained order parameter, \textit{i.e.} by
the functional integral
\begin{equation}
  Z = \int\mathcal{D}\eta \, e^{-\beta L_{\Lambda}[\eta(\vec{r})]},
  \label{eq:part-functional}
\end{equation}
called the Ginzburg-Landau-Wilson (GLW) functional. If we drop the quartic term
in \cref{eq:landau-cg}, then \cref{eq:part-functional} can be solved
straightforwardly by (functional) Gaussian integration. The resulting
\emph{Gaussian approximation} corresponds to the assumption that the
fluctuations of the order parameter are distributed as Gaussian random
variables about the mean value. From the resulting approximation of the free
energy we obtain critical exponents characterizing the phase transition, which
we refer to hereafter as the \emph{mean-field exponents}.

In principle we should be able to improve on the Gaussian approximation by
perturbatively including the quartic term in \cref{eq:landau-cg}. First, we
need to find a small, dimensionless parameter in which to expand. From
\cref{eq:landau-cg} we can define a zero-field ``effective Hamiltonian,"
\begin{equation}
  \ham_{\mathrm{eff}}\sbr{\phi}
  \equiv \beta L_{\Lambda}
  = \int \dif^d \vec{r} \sbr{
    \frac{1}{2}(\nabla\phi)^2 +
    \frac{1}{2} r_0 \phi^2 +
    \frac{1}{4} u_0 \phi^4}
  \label{eq:heff}
\end{equation}
where we have rescaled the order parameter
$\phi \equiv (\beta\gamma)^{1/2}\eta$
so that the coefficient of the derivative term is just $1/2$, and
$r_0/2 \equiv a t/\gamma$,
$u_0/4 \equiv b/(2 \beta \gamma^2)$
\autocite{goldenfeld1992lectures}.
Because $\ham_{\mathrm{eff}}$ is dimensionless, we find from the first term in
\cref{eq:heff} that $\phi$ must have dimensions $L^{1-d/2}$, where $L$ is a
unit of length. Hence we find that $r_0$ has dimensions of $L^{-2}$ and $u_0$
has dimensions of $L^{d-4}$. To write this in terms of dimensionless variables
we use as a length scale $L \equiv r_0^{-1/2}$, which is proportional to the
correlation length in the Gaussian approximation since $\xi \propto t^{-1/2}$.
Thus,
\begin{equation}
  \ham_{\mathrm{eff}}\sbr{\varphi}
  = \int \dif^d \vec{x}
    \sbr{
      \frac{1}{2}
      (\nabla\varphi)^2 +
      \frac{1}{2} \varphi^2 +
      \frac{1}{4} \overline{u}_0 \varphi^4}
\end{equation}
where
\begin{equation}
  \vec{x}\equiv\vec{r}/L\qquad
  \varphi = \phi/L^{1-d/2}\qquad
  \overline{u}_0 = u_0/L^{d-4}.
\end{equation}
Now, to perturbatively include the quartic term, we define
\begin{align}
  \ham_0  &\equiv \int\dif^d\vec{r}\sbr{\frac{1}{2}(\nabla\varphi)^2 + \frac{1}{2} \varphi^2}, \\
  \hamint &\equiv \int\dif^d\vec{r}\sbr{\frac{1}{4} \overline{u}_0 \varphi^4},
  \label{eq:ham-int}
\end{align}
where $\ham_0$ is corresponds to the Gaussian approximation, and expand the
partition function in powers of $\hamint$,
\begin{align*}
  Z
  &= \int\mathcal{D}\phi \, e^{-\ham_{\mathrm{eff}}} \\
  &= \int\mathcal{D}\phi \, e^{-\ham_0} e^{-\hamint} \\
  &= \int\mathcal{D}\phi \, e^{-\ham_0} \del{1 - \hamint + \frac{1}{2} \hamint^2 - \cdots}.
\end{align*}
Such an expansion is reasonable when the coefficient of the quartic term,
$\overline{u}_0$, is small. However, by dimensional analysis we have shown that
$\overline{u}_0 \propto t^{(d-4)/2}$. This means that, in dimensions $d<4$, the
``perturbation" actually diverges as we approach the critical point;
consequently, mean-field theory, including the Gaussian approximation and any
method based on keeping a finite number of terms in the perturbation series,
fails to describe the physics near the critical point. On the other hand, for
$d>4$, the perturbation vanishes in the limit $t \to 0$, and hence including
additional terms in the perturbation series has no effect on the asymptotic
form of the solution. The conclusion is that the mean-field critical exponents
are \emph{exact} in dimensions $d>4$ for systems described by the same Landau
theory as the Ising model (\textit{i.e.}, the Ising universality class). For a
given universality class, we define the \emph{upper critical dimension} $d_u$
to be the dimension above which the mean-field critical exponents are exact.
Hence this result corresponds to $d_u=4$ for the Ising universality class.


\section{Disorder and spin glasses}

Historically, much progress has been made in the theory of condensed matter
physics by studying idealized models in which atoms lie on a regular crystal
lattice. For example, the theory of electronic band structures, which
successfully explains many physical properties of solids, arises from Bloch
waves in a periodic potential.

However, real crystals have some degree of \emph{disorder} in the form of
distortions, impurities, and, at finite temperature, thermal fluctuations of
the atoms about their mean positions. One manifestation of this which is
observed in experiment is a small positive correction to the resistivity of
metals and semiconductors relative to the prediction of band theory. However,
the effect of disorder can be drastic in some cases. In a seminal paper,
\textcite{anderson1958absence} demonstrated the phenomenon now known as
\emph{Anderson localization}, in which the diffusion of electrons is
\emph{entirely} suppressed in three-dimensional (and higher) systems when the
amount of disorder exceeds a certain threshold. Furthermore, in one and two
dimensions, diffusion is suppressed for \emph{any} amount of disorder. The
realization that the presence of disorder may result not just in quantitative
corrections to an idealized theory, but in entirely new qualitative behavior,
motivates the study of disordered systems.

In the early 1970s, experimental studies of magnetic ordering in a class of
materials called ``dilute magnetic alloys" found surprising results. The
materials in question consist of a nonmagnetic ``host" metal (\textit{e.g.} Au,
Ag, Cu, Pt) with a low concentration of magnetic impurities (\textit{e.g.} Fe
or Mn) occupying random sites. One historically-important result concerns the
behavior of such a material at low temperature subjected to an oscillating
magnetic field. Measuring the (ac) susceptibilty, defined as the time average
of $\chi = \partial M / \partial H$, \textcite{cannella1972magnetic} found that
a cusp occured at a specific temperature which depended strongly on the
concentration of impurities. Below the cusp they found no evidence of
ferromagnetic order, excluding the possibility of a ferromagnetic transition.
Neither was this consistent with emergence of \emph{anti}ferromagnetic order,
where we would expect a peak in the susceptibility at the transition
temperature, but this temperature should not to depend strongly on the
concentration of impurities.

A unique feature of the ``classical spin glasses" described above is the
RKKY-type%
\footnote{%
  named for Ruderman, Kittel, Kasuya, and Yosida
}
interaction between the dilute magnetic impurities, mediated by conduction
electrons. Thus, the effective exchange coupling $J(r)$ between two impurity
atoms separated by a distance $r$ has the form
\begin{equation}
  J(r) \sim \frac{\cos(2 k_F r)}{r^3},
\end{equation}
where $k_F$ is the Fermi wavenumber of the host metal
\autocite{binder1986spin}. The important feature here is the oscillation of the
\emph{sign} of the interaction with distance which, when combined with the
spatial disorder of the impurity atoms, leads to interactions between each pair
of impurity atoms which are randomly ferromagnetic or antiferromagnetic. In a
seminal paper which initiated the modern theory of spin glasses,
\textcite{edwards1975theory} identified this competition between ferromagnetic
and antiferromagnetic interactions as the essential physics responsible for the
surprising behavior of spin glasses.


\subsection{Edwards-Anderson model}
\label{sec:intro-ea}

\textcite{edwards1975theory} proposed the following simple-looking model of a
spin glass, whose Hamiltonian has the same form as that of the Ising model,
\begin{equation}
  \ham = -\frac{1}{2} \sum_{ij} J_{ij} S_i S_j - \sum_i H_i S_i,
\end{equation}
but where the nearest-neighbor couplings $J_{ij}$ are independent, Gaussian
random variables with mean zero and variance $\sigma^2$, which is usually taken
to be unity. As for the Ising model, the spins take values $\pm 1$ and are
arranged on a hypercubic lattice in $d$ dimensions.

The Edwards-Anderson (EA) model captures the two essential features of spin
glasses, \emph{frustration} and \emph{quenched disorder}. Frustration means
that there is no configuration of the system which \emph{simultaneously}
minimizes all terms in the energy. A simple example of this is an Ising
antiferromagnet (\textit{i.e.} the Ising model with coupling $J<0$) defined on
a lattice with odd cycle lengths, for example on a hexagonal lattice as shown
in \cref{fig:frust-afm}. Quenched disorder means that the interactions are
random and expected to be independent of time, at least on experimental time
scales. \Cref{fig:frust-disorder} shows how the random interactions in EA model
lead to frustration.

Within the framework of the EA model, we now return to the mystery of the
apparent spin-glass phase transition observed in experiment. In
(anti)ferromagnetism, the broken symmetry is characterized by a scalar order
parameter, the (staggered) magnetization, but this is identically zero in the
EA model. It is not at all obvious what is the nature of the broken symmetry in
the spin glass phase and what is an appropriate order parameter to describe it.
\textcite{edwards1975theory} proposed to use the sum of the squared thermal
averages of the spins,
\begin{equation}
  \qea = \frac{1}{N} \sum_{i=1}^N \av{S_i}^2,
\end{equation}
which vanishes at high temperature, where the behavior is paramagnetic, but is
certainly finite at zero temperature, where the spin settle into some (not
necessarily unique) energy-minimizing configuration. It is now known that, in
dimensions $d>2$, there is a finite-temperature transition to a ``spin-glass
phase" accompanied by a finite value of $\qea$.


It is important to note that it is still unsettled issue whether this
\emph{scalar} order parameter fully characterizes the broken symmetry of the
spin-glass phase. In fact, there are several competing theories which make
quite different predictions as to the nature of the broken symmetry. For
example, ``replica symmetry breaking" (RSB) theory predicts that there are
infinitely many ``pure" states,%
\footnote{%
  That is, states analogous to the ``up" or ``down" states observed in an Ising
  ferromagnet below the transition temperature. See \cref{sec:connection-intro}
  for a discussion of this idea.
}
unrelated by any symmetry, and thus an infinite number of order parameters are
necessary to describe the broken symmetry. In contrast, the ``droplet picture"
predicts that there is only a single pair of pure states, as for the
ferromagnet, and thus a scalar order parameter is sufficient to describe the
broken symmetry. These two theories will be discussed in more detail in
\cref{sec:intro-theories}.

\begin{figure}
  \centering
  \begin{subfigure}{0.49\textwidth}
    \centering
    \includestandalone{figures/frust-afm}
    \subcaption{frustration}\label{fig:frust-afm}
  \end{subfigure}
  \begin{subfigure}{0.49\textwidth}
    \centering
    \includestandalone{figures/frust-disorder}
    \subcaption{frustration and disorder}\label{fig:frust-disorder}
  \end{subfigure}
  \caption[Examples of frustrated Ising systems.]
  {
    Examples of frustrated Ising systems. On the left \subref{fig:frust-afm} is
    an antiferromagnet on a lattice with odd cycles. On the right
    \subref{fig:frust-disorder} is the EA model on a lattice which does not
    necessarily have odd cycles, but will in general have cycles with an odd
    number of antiferromagnetic bonds. In each example there is no assignment
    of ``?" which simultaneously minimizes all terms of the Hamiltonian.
  }
\end{figure}

As alluded to above, despite its apparent simplicity and decades of study, the
EA model remains poorly understood. In contrast to the Ising ferromagnet, which
is found to have a (finite-temperature) phase transition in dimensions $d \geq
2$, the EA model has a finite-temperature transition only for $d>2$.%
\footnote{%
  In fact, evidence suggests that the lower critical dimension is between 2 and
  3 for short-range models \autocite{hartmann2001lower}. Models with long-range
  interactions correspond to short-range models with non-integral effective
  dimension.
}
But computing spin glass partition function, and even finding the ground state,
for $d>2$ appears to be intractable; in fact, both problems have been shown to
be NP-hard by \textcite{barahona1982computational}, suggesting that an eventual
closed-form solution is unlikely. This is unsurprising because the presence of
frustration and disorder means that finding the ground state is a nontrivial
optimization problem.

Even defining a mean-field theory for spin glasses in the
way discussed previously does not seem at first straightforward, since it is
not clear how to deal with the quenched disorder.
\textcite{sherrington1975solvable} made an important contribution to the theory
of spin glasses by proposing a solvable mean-field model of a spin glass. The
solution they provided successfully explained some features observed in
experiment, such as the susceptibility cusp, but opened new important questions
and deepened the mystery in some ways.


\subsection{Sherrington-Kirkpatrick mean-field model}
\label{sec:intro-sk}

As noted above, one way to generate a mean-field theory is to extend
interactions to be infinite-range, so that every degree of freedom interacts
equally with all others, while rescaling the interactions with the inverse
system size to preserve a sensible thermodynamic limit. In this way the model
of \textcite{sherrington1975solvable} is the mean-field counterpart of the EA
model. That is, the Sherrington-Kirkpatrick (SK) model differs from the EA
model in that all pairs of spins interact, with the variance of the couplings
given by $\sigma^2 \propto 1/N$ regardless of distance (by convention the
constant of proportionality is taken to be one).

While the analogous mean-field model has a straightforward solution for the
case of the ferromagnet, the introduction of disorder makes the solution of the
SK model much more difficult. The difficulty arises in carrying out the average
over the quenched disorder to obtain a result which does not depend on the
particular realization of the random couplings. To obtain the disorder-averaged
free energy, formally
\begin{equation}
  \davc{F} \equiv \int \mathcal{D}\JJ \, P(\JJ) F(\JJ)
\end{equation}
(where $\JJ \equiv \cbr{J_{ij}}$ denotes a set of couplings), we need to carry
out a disorder average of the logarithm of the partition function.
\textcite{sherrington1975solvable} did this in clever way, using the so-called
``replica trick," which is based on the identity
\begin{equation}
  \log Z = \lim_{n \to 0} \frac{Z^n-1}{n}.
\end{equation}
Using this to write
\begin{equation}
  -\beta \davc{F}
  = \davc{\log Z}
  = \lim_{n \to 0} \frac{\davc{Z^n}-1}{n},
\end{equation}
we have reduced the problem to computing
\begin{align*}
  \davc{Z^n}
  &=\davc{\prod_{\alpha=1}^n \Tr_{S^{\alpha}}
    e^{-\beta\ham\del{\cbr{S^{\alpha}},\JJ}}} \\
  &=\davc{\Tr_{S^{1}} \cdots \Tr_{S^{n}}
    e^{-\beta\sum_{\alpha=1}^n
    \ham\del{\cbr{S^{\alpha}},\JJ}}} \\
  &=\Tr_{S^{1}} \cdots \Tr_{S^{n}}
    \davc{e^{-\beta\sum_{\alpha=1}^n
    \ham\del{\cbr{S^{\alpha}},\JJ}}},
\end{align*}
for which the disorder average can be computed analytically. Finally, we take
the limit $n \to 0$ to obtain the disorder-averaged free energy. The replica
trick gets its name because the disorder average is done over a compound system
of $n$ replicas, all with the same realization of the disorder $\JJ$, described
by the partition function $Z^n$. Interestingly, the effective Hamiltonian of
the $n$-replica system $\ham_n$, defined by
\begin{equation}
  e^{-\beta\ham_n} =
  \davc{e^{-\beta\sum_{\alpha=1}^n
  \ham\del{\cbr{S^{\alpha}},\JJ}}},
\end{equation}
is not just the sum of the Hamiltonians of the individual replicas; therefore,
the disorder average \emph{couples} the replicas. Because the effective
Hamiltonian is symmetric under the permutation of replicas (after all the
``replicas" were only introduced as a mathematical trick), the solution of
Sherrington and Kirkpatrick makes the (reasonable) assumption that replicas may
be treated on equal footing; theirs is a \emph{replica-symmetric} solution.

The solution given by Sherrington and Kirkpatrick predicts a transition
temperature
\begin{equation}
  \tcmf = \sum_j \davc{J_{ij}^2} \equiv 1,
\end{equation}
and successfully explains some of the features observed in experiment, for
example the cusp in ac susceptibility. However, it soon became clear that it
could not be a correct description of the low-temperature phase, where it was
found to be at odds with simulation results and furthermore predicted an
unphysical \emph{negative} entropy at zero temperature
\autocite{kirkpatrick1978infinite}. An important piece of the puzzle was
provided by \textcite{dealmeida1978stability}, who showed that the
replica-symmetric solution of the SK model becomes unstable for $T<1$ in zero
field.%
\footnote{
  In fact, they show that the replica-symmetric solution is unstable in a
  region of the $T$-$H$ plane for $T<1$ and small fields, thus predicting
  a line of transitions known as the \emph{de Almeida-Thouless line}.
}
The replica-symmetric solution is therefore not observed, just as the
paramagnetic solution of the mean-field Ising model is not observed in the
ordered phase. Instead, the replica symmetry is spontaneously broken.


\subsection{Replica symmetry breaking}
\label{sec:intro-rsb}

The correct \emph{stable} solution of the SK model in the low-temperature phase
was discovered and developed by Giorgio Parisi in a series of seminal papers.%
\footnote{%
  \textcite{%
    parisi1979infinite,
    parisi1980magnetic,
    parisi1980order,
    parisi1983order,
  }
}
The breakthrough idea was that, to find a stable solution in the
low-temperature phase, the replicas can no longer be considered to be
equivalent; although the effective Hamiltonian is symmetric under the
permutation of replicas, this symmetry is spontaneously broken in the
spin-glass phase. More precisely, Parisi considered the \emph{overlap} between
the configurations of two replicas $\alpha$ and $\beta$, defined as
\begin{equation}
  q_{\alpha\beta} = \frac{1}{N} \sum_i \av{S_i^{\alpha} S_i^{\beta}}_n,
\end{equation}
where the average is taken with respect to the $n$-replica effective
Hamiltonian. The replica-symmetric solution of Sherrington and Kirkpatrick
assumes that $q_{\alpha\beta}$ is independent of which replicas $\alpha$ and
$\beta$ are chosen, and is just equal to the Edwards-Anderson order parameter
$\qea$. By contrast, in Parisi's \emph{replica symmetry breaking} (RSB)
solution, the replicas are \emph{not} equivalent; instead, each corresponds to
one of an infinite number of distinct pure states%
\footnote{%
  That is, thermodynamic states in which connected correlations vanish in the
  limit of large distance; the ``up" and ``down" states of the ferromagnetic
  phase are examples of pure states.
}
in the thermodynamic limit; thus $q_{\alpha\beta}$ depends on which $\alpha$
and $\beta$ are chosen, giving the overlap between the corresponding pure
states. The self-overlap of a replica with itself $q_{\alpha\alpha}=\qea$, but
in general $-\qea < q_{\alpha\beta} < \qea$. Note that pure states come in
symmetry-related pairs, so for states $\alpha$, $\beta$, the time-reversed
versions $\overline{\alpha}$, $\overline{\beta}$ are also pure states, and
$q_{\alpha\beta}=-q_{\overline{\alpha}\beta}=q_{\alpha\overline{\beta}}$. Thus
the distribution of the overlap $q$ between two independent equilibrium spin
configurations is an even function, consisting of a superposition of
$\delta$-functions with weights and positions which depend on the disorder
realization $\JJ$,
\begin{equation}
  P_\JJ(q) = \sum_{\alpha\beta} W_{\alpha} W_{\beta} \delta(q - q_{\alpha\beta}),
\end{equation}
where $W_{\alpha}$ and $W_{\beta}$ are the probabilities (``weights") of the
states $\alpha$ and $\beta$ respectively. Now we consider what this looks like
at low temperature, where presumably only a few of the (infinite number of)
pure states have significant weight. The distribution is bimodal, with the
largest weights occurring at $\pm\qea$; these correspond to self-overlaps, and
thus have weight $\sum_{\alpha} W_{\alpha}^2$. Between $-\qea$ and $\qea$ are
smaller peaks corresponding to overlaps of distinct pairs of pure states
$\alpha$, $\beta$ with weights $W_{\alpha} W_{\beta}$. This is sketched in
\cref{fig:pq-rsb-sample}.
\begin{figure}
  \centering
  \begin{subfigure}{0.49\textwidth}
    \centering
    \begin{tikzpicture}
      \pgfmathsetmacro{\x}{0.6}
      \pgfmathsetmacro{\y}{0.9}
      \begin{axis}[
        schematic,
        width=\figurewidth,
        height=0.6\figurewidth,
        xmin=-1,xmax=1,
        ymin=0,ymax=1,
        xtick={-\x,\x},
        xticklabels={$-\qea$,$+\qea$},
        xtick align=outside,
        xlabel={$q$}, ylabel={$P_{\JJ}(q)$},
        enlarge y limits=false,
      ]
        \def\deltapair#1#2{(-#1*\x, #2*\y) (#1*\x, #2*\y)}
        \addplot+[ycomb] coordinates {
          \deltapair{1}{1}
          \deltapair{0.7}{0.4}
          \deltapair{0.3}{0.5}
          \deltapair{0.1}{0.2}
        };
      \end{axis}
    \end{tikzpicture}
    \subcaption{single sample}
    \label{fig:pq-rsb-sample}
  \end{subfigure}
  \begin{subfigure}{0.49\textwidth}
    \centering
    \begin{tikzpicture}
      \pgfmathsetmacro{\x}{0.6}
      \pgfmathsetmacro{\y}{0.9}
      \pgfmathsetmacro{\a}{0.1}
      \pgfmathsetmacro{\b}{0.6}
      \begin{axis}[
        schematic,
        width=\figurewidth,
        height=0.6\figurewidth,
        xmin=-1,xmax=1,
        ymin=0,ymax=1,
        xtick={-\x,\x},
        xticklabels={$-\qea$,$+\qea$},
        xtick align=outside,
        xlabel={$q$}, ylabel={$P(q)$},
        enlarge y limits=false,
      ]
        \addplot+[ycomb,forget plot] coordinates {(-\x,\y) (\x,\y)};
        \addplot+[domain=-\x:\x] {\a + \b*x^2};
      \end{axis}
    \end{tikzpicture}
    \subcaption{disorder average}
    \label{fig:pq-rsb-dav}
  \end{subfigure}
  \caption[
    Sketches of the overlap distribution for a single sample and the sample
    average, according to replica symmetry breaking (RSB) theory.
  ]
  {
    Sketches of the overlap distribution for \subref{fig:pq-rsb-sample} a
    single sample and \subref{fig:pq-rsb-dav} the sample average, according to
    replica symmetry breaking (RSB) theory. The support in the region
    $-\qea<q<\qea$ in the disorder-averaged $P(q)$ in \subref{fig:pq-rsb-dav}
    comes from the many pairs of pure states for each sample, \textit{e.g.} in
    \subref{fig:pq-rsb-sample}.
  }
\end{figure}

One surprising result of RSB theory is that the overlap distribution for a
single disorder realization $P_\JJ(q)$ is non-self-averaging, meaning that it
depends on $\JJ$ even in the thermodynamic limit. This is at first a little
unsettling, because the whole procedure of carrying out the disorder average is
justified by the thermodynamic assumption that we can determine the properties
of a macroscopic system as an average over subsystems. But this turns out not
to be contradictory, as all \emph{observable} properties are self-averaging
within the RSB framework \autocite{stein2013spin}. Carrying out the disorder
average, we obtain an overlap distribution that has two $\delta$-function peaks
at $\pm\qea$ and is finite everywhere in between. This is sketched in
\cref{fig:pq-rsb-dav}. The weight of the sample-averaged distribution at $q=0$
is found to be a constant independent of system size for large $L$.

The RSB solution appears to be the correct description of the low-temperature
phase of the SK model, and several of its predictions (although notably not the
non-self-averaging of the overlap distribution) have more recently been proven
rigorously for the SK model \autocite{talagrand2003spin}. This has led Parisi
and others to conjecture that RSB also provides an accurate description
spin-glass phase in more realistic, short-range models such as the
Edwards-Anderson model. This is at odds with other competing theories of the
spin-glass phase in short-range models.


\subsection{Theories of short-range spin glasses}
\label{sec:intro-theories}

The RSB solution of the Sherrington-Kirkpatrick mean-field model predicts
several surprising features, such as non-self-averaging of the order parameter
distribution and the existence of a line of transitions in nonzero field
(\textit{i.e.} the de Almeida-Thouless line), that are not observed in systems
without disorder. It has been a controversial issue whether these exotic
features, which certainly exist in the SK model, persist in more realistic
models with short-range interactions, or whether they are artifacts of the
infinite-range interactions, and the nature of the spin-glass phase in
short-range models can be explained by a simpler phenomenological picture.

One well-known alternate scenario is the ``droplet picture" (DP) introduced by
Fisher and Huse
\autocite{fisher1986ordered,fisher1987absence,fisher1988equilibrium},
based on the phenomenological scaling arguments of
\textcite{mcmillan1984scaling}. The central \textit{ansatz} is that
lowest-energy excitations of a ground state which have spatial extent $L$ have
an energy cost which scales as $L^{\theta}$, where $\theta$ is a positive
(``stiffness") exponent. Consequently, in the thermodynamic limit, excitations
which flip a finite fraction of the spins cost an infinite amount of energy.
Additionally, the excitations are predicted to have a fractal surface, with
dimension $d_s<d$. Both of these statements are at odds with the predictions of
RSB, according to which excitations have an energy which is independent of $L$
and hence excitations which flip a finite fraction of spins can have
\emph{finite} energy cost in the thermodynamic limit; also, excitations in RSB
are predicted to be space-filling, \textit{i.e.} to have a dimension $d_s=d$.

Further analysis reveals that, in the droplet picture, there can be only a
single time-reversed pair of pure states in the thermodynamic limit, in
contrast to the infinitely-many predicted by RSB theory. Hence the droplet
picture predicts an overlap distribution which is sharply different from the
prediction of RSB in the thermodynamic limit. As discussed previously, RSB
theory predicts finite probability everywhere between $-\qea$ and $\qea$. In
contrast, the droplet picture predicts finite probability in the region
$-\qea<q<\qea$ only for finite systems; $P(q)$ in this region is expected to
vanish as $L^{-\theta}$ \autocite{moore1998evidence} in the thermodynamic
limit, leaving a trivial distribution with two delta functions at $\pm\qea$ and
vanishing weight in between. Both distributions are sketched in
\cref{fig:pq-droplet-rsb}.

Because $P(q)$ is relatively easy to measure in simulations, many numerical
studies%
\footnote{%
  \textit{e.g.}
  \textcite{%
    marinari2000replica,
    reger1990monte,
    katzgraber2001monte,
    katzgraber2003monte,
  }
}
have attempted to find evidence for one picture or the other by studying the
overlap distribution near $q=0$ for a range of system sizes $L$ and looking for
a $L^{-\theta}$ decay as predicted by the droplet picture. The results appear
consistent with RSB in that $P(0)$ varies only very slowly with $L$; however,
it has been argued that the sizes accessible to simulation are too small to
observe the asymptotic behavior, and the droplet picture cannot be ruled out
\autocite{moore1998evidence,middleton2013extracting}. In \cref{chap:overlap}
of this thesis we study alternate statistics of the overlap distributions
of individual samples, $P_{\JJ}(q)$, in an attempt to distinguish the behavior
as RSB-like or droplet-like.

\begin{figure}
  \centering
  \begin{subfigure}{0.49\textwidth}
    \centering
    \pgfmathsetmacro{\x}{0.6}
    \pgfmathsetmacro{\y}{0.9}
    \begin{tikzpicture}
      \begin{axis}[
        schematic,
        width=\figurewidth,
        height=0.6\figurewidth,
        xmin=-1, xmax=1,
        ymin=0, ymax=1,
        xtick={-\x,\x},
        xticklabels={$-\qea$,$+\qea$},
        xtick align=outside,
        ytick=\empty,
        xlabel={$q$}, ylabel={$P(q)$},
      ]
        \addplot+ [ycomb] coordinates {(-\x,\y) (\x,\y)};
      \end{axis}
    \end{tikzpicture}
    \subcaption{droplet picture}
    \label{fig:pq-droplet}
  \end{subfigure}
  \begin{subfigure}{0.49\textwidth}
    \centering
    \begin{tikzpicture}
      \pgfmathsetmacro{\x}{0.6}
      \pgfmathsetmacro{\y}{0.9}
      \pgfmathsetmacro{\a}{0.1}
      \pgfmathsetmacro{\b}{0.6}
      \begin{axis}[
        schematic,
        width=\figurewidth,
        height=0.6\figurewidth,
        xmin=-1,xmax=1,
        ymin=0,ymax=1,
        xtick={-\x,\x},
        xticklabels={$-\qea$,$+\qea$},
        xtick align=outside,
        xlabel={$q$}, ylabel={$P(q)$},
        enlarge y limits=false,
      ]
        \addplot+[ycomb,forget plot] coordinates {(-\x,\y) (\x,\y)};
        \addplot+[domain=-\x:\x] {\a + \b*x^2};
      \end{axis}
    \end{tikzpicture}
    \subcaption{RSB picture}
    \label{fig:pq-rsb}
  \end{subfigure}
  \caption[
    Comparison of disorder-averaged overlap distributions in the thermodynamic
    limit for the droplet picture and the RSB picture.
  ]
  {
    Comparison of disorder-averaged overlap distributions in the thermodynamic
    limit for \subref{fig:pq-droplet} the droplet picture and
    \subref{fig:pq-rsb} the RSB picture. The droplet $P(q)$ is trivial in the
    thermodynamic limit, with two delta functions at $\pm\qea$. In contrast,
    the RSB $P(q)$ has finite weight everywhere in the region $-\qea<q<\qea$.
  }
  \label{fig:pq-droplet-rsb}
\end{figure}

\textcite{newman1996non} lend some support for the argument that RSB is not a
realistic description of short-range spin glasses with a proof that
non-self-averaging \emph{cannot} occur based on simple symmetry considerations.
Thus the ``standard" RSB scenario, in which the single-sample overlap
distribution is non-self-averaging, cannot be correct for short-range spin
glasses. Newman and Stein (NS) argue that a flaw lies in the assumption of
convergence to a unique mixed state [\textit{i.e.} a unique distribution
$P_{\JJ}(q)$] in the thermodynamic limit; instead, they argue, the mixed state
which is observed has the property of \emph{chaotic size dependence}
\autocite{newman1992multiple}, meaning that it wanders continually with
increasing $L$, sampling from all possible mixed states. The fundamental
object, then, is the probability distribution which is sampled, called the
``metastate." In \cref{chap:connection} we investigate the connection between
the metastate and the sampling of pure states generated by nonequilibrium
dynamics.

Within the metastate description, \textcite{newman1997metastate} show that it
is possible to recover the many-states picture of RSB, \textit{i.e.} an overlap
distribution $P(q)$ like that shown in \cref{fig:pq-rsb}. In this
``non-standard RSB" (which is actually the only viable form of RSB for
short-range spin glasses), the disorder-averaged overlap distribution of
standard RSB theory is instead obtained from a metastate average for a
\emph{single} disorder realization. Thus the NS description avoids the property
of non-self-averaging which is the downfall of the standard RSB picture applied
to short-range spin glasses.

While NS have provided an interpretation of RSB which is viable for short-range
spin glasses, results concerning the invariance of the metastate under a change
of boundary conditions strongly suggest that it cannot occur in realistic spin
glasses, and instead a simpler picture (which is still consistent with chaotic
size dependence) holds \autocite{newman1998simplicity}. One such proposal is
the \emph{chaotic pairs} picture \autocite{newman1996non} which, like the
droplet picture, predicts a trivial structure for the sample-averaged overlap
distribution, with a pair of delta function peaks at $\pm\qea$ (\textit{i.e.}
the same as \cref{fig:pq-droplet}). However, in contrast to the droplet
picture, in the chaotic pairs picture this distribution arises from a pair of
pure states which exhibit chaotic size dependence but whose \emph{overlaps}
converge to $\pm\qea$ in the thermodynamic limit.

\section{Organization of the dissertation}

  In \cref{chap:numerical} we give an overview of the numerical methods used in
  the research.

  In \cref{chap:nonextensive}, we study one-dimensional long-range (1DLR) spin
  glass models to fill in a previously unexplored region of parameter space in
  which the interactions become sufficiently long-range that they must be
  rescaled with the system size to maintain the thermodynamic limit. We find
  strong evidence that detailed behavior of the 1DLR models everywhere in this
  ``nonextensive regime" is identical to that of the Sherrington-Kirkpatrick
  model, lending support to a recent conjecture.

  In \cref{chap:overlap} we attempt to distinguish the RSB and droplet pictures
  by studying recently-proposed observables based on the statistics of
  individual disorder samples, rather than simply averaging over the disorder
  as is most frequently done in previous studies. We compare Monte Carlo
  results for 1DLR models which are proxies for short-range models in 3, 4, and
  10 dimensions with previously-obtained data for the 3D and 4D short-range
  models and the SK model. For one statistic, which is expected to sharply
  distinguish between the two pictures in the thermodynamic limit, we find that
  larger system sizes than those currently feasible to simulate are needed to
  obtain an unambiguous result. We also find that two other recently-proposed
  statistics, the median of the cumulative overlap distribution and the
  ``typical" overlap distribution, are not particularly helpful in
  distinguishing between the RSB and droplet pictures.

  In \cref{chap:connection} we study the evolution of dynamical correlations in
  a 1DLR model which is a proxy for a short-range model in $d=8$ dimensions. We
  find that the spatial decay of the correlations at distances less than the
  dynamical correlation length $\xi(t)$ agrees quantitatively with the
  predictions of the metastate theory, evaluated according to the RSB picture.
  We also compute the dynamic exponent defined by $\xi(t) \propto t^{1/z(T)}$
  and find that it is compatible with the mean-field value of the critical
  dynamical exponent for short-range spin glasses.

  Finally, in \cref{chap:fss} we present a unified view of finite-size scaling
  (FSS) in dimensions $d$ above the upper critical dimension $d_u$, for both
  free and periodic boundary conditions. For $d>d_u$, a dangerous irrelevant
  variable is responsible for both the violation of hyperscaling and the
  violation of ``standard" FSS. We find that the modified hyperscaling proposed
  to allow for this applies only to $\vec{k}=\vec{0}$ fluctuations, while
  standard FSS applies to $\vec{k}\neq\vec{0}$ fluctuations. Hence the exponent
  $\eta$ describing the power-law decay of correlations at criticality is
  unambiguously $\eta=0$. With free boundary conditions, the finite-size
  ``shift" is greater than the rounding. Nonetheless, using $T-T_L$, where
  $T_L$ is the finite-size pseudocritical temperature, as the scaling variable,
  the data do collapse onto a scaling form that includes the behavior both at
  $T_L$, where the susceptibility $\chi$ diverges like $L^{d/2}$, and the bulk
  $T_c$, where it diverges like $L^2$. We support these claims with data from
  large-scale simulations of the five-dimensional Ising model.

%Trivial-nontrivial \autocite{krzakala2000spin,palassini1999triviality}

%In the vicinity of a continuous phase transition it is found experimentally
%that various measurable quantities depend as a power law on the distance from
%the critical point. For example, from Onsager's exact solution of the
%two-dimensional Ising model, the magnetization $M \propto (T_c-T)^{1/8}$ as we
%approach $T_c$ from below, and the correlation length diverges like $\xi \propto
%(T-T_c)^{-7/4}$ as we approach $T_c$ from above.

%\begin{table}
%  \centering
%  \begin{tabular}{>{$}r<{$}l>{$}r<{$}>{$}r<{$}>{$}r<{$}>{$}r<{$}}
%    \toprule
%    \text{Exponent} & \text{Related quantity} & \text{MFT} & \text{Experiment} &
%    \text{Ising ($d=2$)} & \text{Ising ($d=3$)} \\
%    \midrule
%    \alpha & specific heat         & 0   & 0.110 \text{--} 0.116 & 0   & 0.110(5) \\
%    \beta  & order parameter       & 1/2 & 0.316 \text{--} 0.327 & 1/8 & 0.325  \pm 0.0015 \\
%    \gamma & susceptibility        & 1   & 1.23  \text{--} 1.25  & 7/4 & 1.2405 \pm 0.0015 \\
%    \delta & critical isotherm     & 3   & 4.6   \text{--} 4.9   & 15  & 4.82(4) \\
%    \nu    & correlation length    & 1/2 & 0.625 \pm 0.010       & 1   & 0.630(2) \\
%    \eta   & critical correlations & 0   & 0.016 \text{--} 0.06  & 1/4 & 0.032  \pm 0.003 \\
%    \bottomrule
%  \end{tabular}
%  \caption[Summary of critical exponents for the Ising universality class.]
%  {
%    \textit{Adapted from \textcite{goldenfeld1992lectures}.} Summary of
%    critical exponents for the Ising universality class. The experimental
%    values quoted are obtained from experiments on fluid systems, yet are
%    consistent with numerical results for the three-dimensional Ising model.
%  }
%\end{table}

