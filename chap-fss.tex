\chapter{Finite-size scaling above the upper critical dimension}
\label{chap:fss}

\section{Introduction}
\label{sec:fss-intro}

The theory of finite-size scaling (FSS) bridges the gap between the critical
behavior of \emph{finite} systems and that of infinite (or effectively
infinite) systems which are commonly studied in analytical theory and
experiment. As such, FSS is ubiquitous in the literature of computational
physics, where it is used extensively to extrapolate \emph{bulk} (\emph{i.e.}
$L\to\infty$) behavior, which can be compared with analytical or experimental
results, from the results of numerical simulation of finite systems
\autocite{binder2001monte}.
% TODO: awkward ^

As discussed in \cref{sec:numerical-fss}, the central assumption of
\emph{standard} FSS is that finite-size corrections only involve the ratio of
the system size $L$ to the bulk (\emph{i.e.} infinite system size) correlation
length $\xi$.
% TODO: ref for this?
While this assumption turns out to be correct in dimensions $d$ less than the
upper critical dimension $d_u$, the situation is more complicated \emph{above}
the upper critical dimension, $d>d_u$. This is a bit surprising since for
$d>d_u$ the critical exponents are independent of $d$ and are predicted exactly
by mean-field theory. Thus, we might naively expect from
\cref{eq:standard-fss-susc} that, for $d>d_u$, the susceptibility scales with
$L$ like
\begin{equation}
  \chi(L,T) \sim L^2 \scalefunc{\chi}\sbr{L^2 \del{T-T_c}},
  \label{eq:fss-above-naive}
\end{equation}
where $\scalefunc{\chi}$ is a scaling function and we have inserted the
mean-field exponents $\gamma=1$ and $\nu=1/2$. Unfortunately, it is not that
simple, and instead it turns out that the basic assumption of FSS, that $L$
dependence enters only through the ratio $L/\xi$, is invalid for $d>d_u$. The
trouble is related to the observation that, for $d>d_u$, ``hyperscaling"
relations such as $d \nu = \gamma - 2\beta$ are necessarily violated, since the
critical exponents ``stick" at their mean-field values for all values of
$d>d_u$. It turns out that a ``dangerous irrelevant variable" is responsible
for both effects. To understand this we need a more sophisticated approach
based on the renormalization group (RG).

According to renormalization-group derivations of FSS
\autocite{privman1983finite}, the singular part of the free energy $f_L$ and
the correlation length $\xi_L$ have the form
\begin{align}
  f_L &= L^{-d} \scalefunc{f}\del{t L^{y_t}, h L^{y_h}, u L^{y_u}}, \\
  \xi_L &= L \scalefunc{\xi}\del{t L^{y_t}, h L^{y_h}, u L^{y_u}},
\end{align}
where $t\equiv(T-T_c)/T_c$ is the reduced temperature, $h$ is the magnetic
field, and $u$ is the quartic coupling of the Landau theory; $y_t$, $y_h$, and
$y_u$ are the corresponding renormalization-group exponents.
For $d < d_u$, $u$ is a relevant variable ($y_u > 0$, see \cref{sec:intro-landau})
and for $h=0$ we obtain the scaling forms
\begin{subequations}
\begin{align}
  \chi(L,t) &= L^{\gamma y_t} \scalefunc{\chi}\del{t L^{y_t}}
  \label{eq:fss-standard-susc} \\
  g(L,t) &= \scalefunc{\chi}\del{t L^{y_t}},
\end{align}
\label{eq:fss-standard}
\end{subequations}
where $y_t=1/\nu$, for the susceptibility and Binder ratio [defined in
\cref{eq:binder}] respectively.

For $d > d_u$, $u$ is irrelevant ($y_u<0$), but the corresponding derivation is
complicated by the fact that the scaling function $\scalefunc{f}(x,y,z)$ is
singular in the limit $z \to 0$, \textit{i.e.}, $u$ is a \emph{dangerous
  irrelevant variable}. Therefore we can't simply substitute $z=0$ in the
scaling function, and must instead evaluate the limit $z \to 0$, assuming a
particular form of the singularity.


\subsection{Periodic boundary conditions}

For $k=0$ fluctuations%
\footnote{%
  \emph{i.e.}, fluctuations in the $k=0$ mode of the order parameter, for
  example in the (uniform) magnetization, $\sum_i S_i$.
}
in systems with periodic boundary conditions, \textcite{binder1985finite} show
that, for $d>d_u$, the thermal exponent $y_t$ is replaced by $y_t^*$,
\begin{subequations}
\begin{align}
  \chi(L,t) &= L^{y_t^*} \scalefunc{\chi}\del{L^{y_t^*} t},
  \label{eq:fss-above-susc} \\
  g(L,t) &= \scalefunc{\chi}\del{L^{y_t^*} t},
  \label{eq:fss-above-binder}
\end{align}
\label{eq:fss-above}
\end{subequations}
where
\begin{equation}
  y_t^*=d/2.
  \label{eq:fss-above-yt}
\end{equation}
% TODO: underlying assumption gives correct bulk behavior in all limiting cases
This is a surprising result because it predicts that finite-size corrections
appear not when $\xi \sim L$, as is assumed in standard FSS, but rather only
when $\xi \sim L^{d/4}$, a length scale larger than the size of the system.%
\footnote{
  To see this, note that for $d > d_u$, $t \sim \xi^{-1/\nu} = \xi^{-2}$, so
  the argument of the scaling function $L^{y_t^*} t = L^{d/2} \xi^{-2}$ is of
  order unity when $\xi \sim L^{d/4}$.
}
Consequently, the finite-size transition is ``rounded out" over a temperature
range which scales like $L^{-d/2}$, smaller than $L^{-2}$ predicted by standard
FSS.

An extensive set of works%
\footnote{%
  For example,
  \textcite{
    luijten1996finite,
    parisi1996scaling,
    blote1997universality,
    luijten1999finite,
    binder2001monte,
  }
}
have shown the validity of \cref{eq:fss-above}, though it required large system
sizes, good statistics, and appreciation that \emph{corrections} to FSS are
large and slowly decaying for the range of sizes that can feasibly be
simulated.
%The (universal) value $\scalefunc{g}(0)$, computed by
%\textcite{brezin1985finite} (who showed it to be simply that obtained by
%including \emph{only} the $k=0$ mode with $T_c$ adjusted to the correct value)
%has been confirmed.
% TODO: wording/correctness ^
% TODO: understand this ^


\subsection{Free boundary conditions}

As stated above, \cref{eq:fss-above} makes the rather surprising prediction
that, for $d>d_u$ and periodic boundary conditions, finite-size effects set in
not when $\xi \sim L$, but even closer to criticality, when $\xi \sim L^{d/4}$.
It is therefore interesting to ask what is the corresponding behavior for free
boundary conditions, where we expect that \emph{something} must happen when the
correlation length $\xi \sim L$ \autocite{jones2005finite}. In fact,
\textcite{rudnick1985effect} have argued analytically that a temperature
\emph{shift} of order $L^{-2}$ has to be included with free boundary
conditions, in addition to the rounding of order $L^{-d/2}$.

To explain this, note that the exponents $y_t$ in \cref{eq:fss-standard} and
$y_t^*$ in \cref{eq:fss-above} are ``rounding" exponents since they control the
temperature range over which a singularity is rounded out. To define the
``shift" exponent, we first define, for each size $L$, a ``finite-size
pseudocritical temperature" $T_L$ by, for example, the location of the peak in
some susceptibility, or the temperature at which the Binder ratio [defined in
\cref{eq:binder}] has a particular value. The difference $T_c-T_L$ goes to zero
for $L\to\infty$ like
\begin{equation}
  T_c - T_L = \frac{A}{L^{\lambda}},
  \label{eq:TL-scaling}
\end{equation}
defining the shift exponent $\lambda$. The precise value of $T_L$ depends on
which criterion is used to define it, but the exponent $\lambda$ is expected to
be independent of the definition. Whether or not the amplitude $A$ depends on
the quantity used to define the shift will be discussed in
\cref{sec:fss-results-fbc}.
% TODO: redundant/contradictory? ^

If $\lambda$ is less than the rounding exponent, which will turn out to be the
case for free boundary conditions, then the shift is \emph{larger} than the
rounding, so we need to modify \cref{eq:fss-above} to
\begin{subequations}
\begin{align}
  \chi(L,t) &= L^{y_t^*} \scalefunc{\chi}\sbr{L^{y_t^*}\del{T-T_L}},
  \label{eq:fss-above-shift-susc} \\
  g(L,t) &= \scalefunc{\chi}\sbr{L^{y_t^*}\del{T-T_L}},
  \label{eq:fss-above-shift-binder}
\end{align}
\label{eq:fss-above-shift}
\end{subequations}
in which the argument of the scaling function involves the difference between
$T$ and the pseudocritical temperature $T_L$. We verify
\cref{eq:fss-above-shift} in
\cref{fig:qshift,fig:qwidth,fig:qchiTL,fig:chi-f-scaling} below.

The criterion that the shift is given by the (standard FSS) condition $\xi \sim
L$ yields $\lambda=2$, as proposed by \textcite{rudnick1985effect} and
confirmed in simulations by \textcite{berche2012hyperscaling,kenna2013new}. As
with \cref{eq:fss-standard-susc} we must have $\scalefunc{\chi}(x) \propto
x^{-1}$ for $x \to \infty$ in order to recover the correct bulk behavior above
$T_c$. Setting $T=T_c$, we have
\begin{equation}
  \chi(L,T_c) = L^{d/2}\scalefunc{\chi}(A L^{d/2-2}),
\end{equation}
and therefore, asymptotically for large $L$,
\begin{equation}
  \chi(L,T_c) \propto L^2
  \quad\text{(free, $k=0$)},
  \label{eq:rounding-k0-Tc}
\end{equation}
a result that has been shown rigorously.
% TODO: reference
Hence, in contrast to \textcite{berche2012hyperscaling}, we propose that the
region at the bulk $T_c$ is part of the scaling function.
% TODO: explain "part of"
Similarly, for the Binder ratio, $\scalefunc{g}(x) \propto x^{-2}$
% TODO: why?
for $x\to\infty$, which gives
\begin{equation}
  g(L,T_c) \propto \frac{1}{L^{d-4}}
  \quad\text{(free, $k=0$)}.
  \label{eq:binder-scaling-Tc}
\end{equation}
With periodic boundary conditions, the intersection of the data for $g$ for
different sizes provides a convenient estimate of $T_c$, but, as
\cref{eq:binder-scaling-Tc} shows, this method cannot be used for free boundary
conditions because $g$ vanishes at $T_c$ for $L\to\infty$. In fact, we will see
from the data in \cref{sec:fss-results-fbc} that there are no intersections at
all. However, we will not be able to verify the precise form in
\cref{eq:binder-scaling-Tc} because the values of $g$ at $T_c$ are very small,
below the noise threshold of our simulations.

So far we have discussed only $k=0$ fluctuations. However, it is also necessary
to discuss fluctuations at $\vec{k}\neq\vec{0}$, since we need these to
determine the spatial decay of the correlation functions.
% TODO: explain
Of particular importance is the decay of the correlations at $T_c$, which fall
off with distance like $1/r^{d-2+\eta}$, where the mean-field value of the
$\eta$ exponent is zero.
% TODO: mention anomalous dimension?
In the mean-field regime ($d>d_u$), the fluctuations of the $k=0$ modes are
Gaussian, so the Binder ratio is always zero.
% TODO: explain
For the wavevector-dependent susceptibility, we will argue that \emph{standard}
FSS, \cref{eq:fss-standard-susc}, holds for both boundary conditions (BCs),
\emph{i.e.}
\begin{equation}
  \chi(\vec{k},L,T) =
  L^2 \scalefunc{\chi}\sbr{L^2 \del{T-T_c}, kL}
  \quad\text{(both BCs, $\vec{k}\neq\vec{0}$)},
  \label{eq:susc-kn-scaling}
\end{equation}
where we have put the explicit $k$ dependence in a natural way as a second
argument of the scaling function. We note that \cref{eq:susc-kn-scaling} holds
for the spherical model%
\footnote{%
  Eq.~(22) of \textcite{shapiro1986fully} and Eq.~(37) of
  \textcite{brezin1982investigation} correspond to
  \cref{eq:susc-kn-scaling} with
  $\scalefunc{\chi}(x,y)=\del{x^2+y^2}^{-1}$, at least above $T_c$.
}
with periodic boundary conditions.%
\footnote{%
  The spherical model, in which the length constraint $S_i^2=1$ on each spin is
  replaced by a \emph{single} global average constraint, is equivalent to an
  $n$-component vector model in the limit of $n\to\infty$ for the case of
  periodic boundary conditions. For free boundary conditions, however, the
  correspondence does not hold. In that case, due to a lack of translational
  invariance, one would need a \emph{different} average constraint on each spin
  to reproduce the results of the vector model with an infinite number of
  components.
}
For free boundary conditions, the Fourier modes are not plane waves (see
\cref{sec:fss-model}), and, by $\vec{k}\neq\vec{0}$, we really mean modes that
are orthogonal to the uniform ($k=0$) magnetization and thus do not develop a
nonzero expectation value below $T_c$.

If we fix $T=T_c$ in \cref{eq:susc-kn-scaling} and consider $k L \gg 1$, then
the size dependence must drop out, so $\scalefunc{\chi}(0,y) \propto y^{-2}$
and therefore
\begin{equation}
  \chi(\vec{k},L,T_c) \propto \frac{1}{k^2}
  \quad\text{($k L \gg 1$)}.
  \label{eq:susc-kn-Tc-scaling}
\end{equation}
How, then, do correlations fall off in real space at criticality? To fully
understand this, we have to consider separately the contribution from the $k=0$
mode, as in Bose-Einstein condensation.
% TODO: explain
If $C(\vec{r})$ is the spin-spin correlation function at displacement $\vec{r}$
and $\widetilde{C}(\vec{k})$ is the Fourier transform (FT), then, as shown in
\cref{eq:susc-kn-scaling},
\begin{equation}
  \widetilde{C}(\vec{k}) \propto \frac{1}{k^2}
  \label{eq:ft-corr-kn}
\end{equation}
for $k \to 0$. However, for $k=0$ we note that $\widetilde{C}(k=0) =
\chi(L,T)/L^d$, see \cref{eq:susc-k0} below, and from \cref{eq:fss-above-susc}
this gives
\begin{equation}
  \widetilde{C}(k=0) \propto \frac{1}{L^{d/2}}.
\end{equation}
The real-space correlation function at distance $L/2$ is then given by the FT
\begin{equation}
  C(\vec{\hat{z}}L/2) =
  \del{\frac{L}{2\pi}}^d
  \int_{\vec{k}\neq\vec{0}}\dif^d k\,
  \widetilde{C}(\vec{k})\exp\del{i\vec{k} \cdot L\vec{\hat{z}}/2}
  + \widetilde{C}(k=0).
  \label{eq:rs-corr}
\end{equation}
% TODO: what is zhat?
Using \cref{eq:ft-corr-kn}, which correctly gives the FT at large $r$, the
first term in \cref{eq:rs-corr} is proportional, on dimensional grounds, to
$1/L^{d-2}$. This is smaller than the second term, which is proportional to
$1/L^{d/2}$. Thus, $C(\vec{\hat{z}}L/2) \propto 1/L^{d/2}$, in agreement with
Fig.~1 of \textcite{kenna2014fisher}. Nonetheless, correlations fall off with
distance like $1/r^{d-2}$. The resolution of this apparent discrepancy is that
the $k=0$ mode has to be treated separately and gives the dominant contribution
to $C(\vec{\hat{z}}L/2)$. We therefore do not see the need for the second
$\eta$-like exponent proposed by \textcite{kenna2014fisher}.
% TODO: reword/expand on this paragraph ^

While \cref{eq:susc-kn-scaling} does not seem to have been stated in the
literature before, to our knowledge, it is actually quite natural. The
dangerous irrelevant variable, which is the quartic coupling in the
Ginzburg-Landau-Wilson effective Hamiltonian, is needed to control the
expectation value of the ($k=0$) order parameter, which leads to nonstandard
FSS for $k=0$ fluctuations. However $\vec{k}\neq\vec{0}$ fluctuations (more
precisely, fluctuations that do not acquire a nonzero expectation value) are
not directly affected by the dangerous irrelevant variable, and consequently
they have standard FSS.

% TODO: plan of paper?

\section{Model}
\label{sec:fss-model}

We consider an Ising model in $d=5$ dimensions in zero field, described by the
Hamiltonian
\begin{equation}
  \ham = -\frac{1}{2} \sum_{ij} J_{ij} S_i S_j,
\end{equation}
where the $J_{ij}=1$ if $i$ and $j$ are nearest neighbors and zero otherwise,
and the spins $S_i$ take values $\pm 1$. Previous simulations have determined
the transition temperature very precisely, finding
\begin{equation}
  T_c = 8.77846(3)
  \label{eq:Tc-d5}
\end{equation}
\autocite{luijten1999finite}. We simulate the model efficiently using the Wolff
cluster algorithm described in \cref{sec:numerical-cluster}, with which we can
study sizes up to $L=64$ (which has about a billion spins).
% TODO: L=64 results included?

We calculate various moments of the uniform magnetization per spin,
\begin{equation}
  m = \frac{1}{L^d} \sum_{i=1}^N S_i,
  \label{eq:mag-k0}
\end{equation}
including the uniform susceptibility%
\footnote{%
  This expression differs from the standard expression for the susceptibility
  $\chi=\beta L^d \del{\av{m^2}-\av{m}^2}$ in two ways. The first is that we
  omit the factor of $\beta$, which is conventional in studies of critical
  phenomena. Secondly, and less trivially, we ignore the subtracted term, which
  is hard to compute reliably in Monte Carlo simulations since one would have
  to apply a field $h$ (to break the symmetry) and take the limit $h \to 0$
  \emph{after} the limit $L\to\infty$. Thus the quantity we call $\chi$ is
  really only the susceptibility for $T>T_c$. It is, nonetheless, a convenient
  quantity to study, and it has the claimed scaling behavior.
}
\begin{equation}
  \chi = L^d \av{m^2}
  \label{eq:susc-k0}
\end{equation}
and the Binder ratio
% TODO: ref intro description of Binder ratio?
\begin{equation}
  g = \frac{1}{2}\del{3 - \frac{\av{m^4}}{\av{m^2}^2}}.
  \label{eq:binder}
\end{equation}
In addition, we compute the wavevector-dependent susceptibilities
\begin{equation}
  \chi(\vec{k}) = L^d \av{\abs{m(\vec{k})}^2},
\end{equation}
in which the wavevector-dependent magnetization, $m(\vec{k})$, is defined
differently for periodic and free boundary conditions as follows.

For periodic boundary conditions, the Fourier modes are plane waves,
so we have
\begin{equation}
  m(\vec{k}) = \frac{1}{N} \sum_i e^{i\vec{k}\cdot\vec{r}} S_i
  \quad\text{(periodic)},
\end{equation}
where
\begin{equation}
  k_{\alpha} = 2\pi n_{\alpha}/L
  \quad\text{(periodic)},
\end{equation}
where $n_{\alpha}\in\cbr{0,1,\dots,L-1}$ and $\alpha$ denotes a Cartesian
coordinate.

For free boundary conditions, the Fourier modes are sine waves,
\begin{equation}
  m(\vec{k}) = \frac{1}{N} \sum_i
  \sbr{\prod_{\alpha=1}^d \sin\del{k_{\alpha},r_{i,\alpha}}} S_i
  \quad\text{(free)},
  \label{eq:modes-free}
\end{equation}
where
\begin{equation}
  k_{\alpha} = \pi n_{\alpha}/(L+1)
  \quad\text{(free)},
  \label{eq:allowed-free}
\end{equation}
where $n_{\alpha}\in\cbr{1,2,\dots,L}$ and the components of the lattice
position $r_{i,\alpha}$ also take values between 1 and $L$. There is zero
contribution to the sum in \cref{eq:modes-free} if we set $r_{i,\alpha}=0$ or
$L+1$, so \cref{eq:modes-free,eq:allowed-free} correctly incorporate free
boundary conditions.

Note that $k=0$ is not an allowed mode with free boundary conditions, so the
uniform magnetization in \cref{eq:mag-k0} does not correspond to a single
Fourier mode in this case. Note also that modes with \emph{all} $n_{\alpha}$
odd have a projection onto the uniform magnetization and so will acquire a
nonzero expectation value below $T_c$ in the thermodynamic limit. Such modes
will therefore be subject to the nonstandard FSS in \cref{eq:fss-above}.
However, if any of the $n_{\alpha}$ are even, there is no projection onto the
uniform magnetization, so they will not acquire an expectation value below
$T_c$ and will therefore be subject to the standard FSS in
\cref{eq:susc-kn-scaling}.


\section{The quotient method}
\label{sec:fss-quotient}

The discussion in \cref{sec:fss-intro} assumed that the sizes are
sufficiently large and $T$ sufficiently close to $T_c$ that corrections to FSS
are negligible. For free boundary conditions, however, a substantial fraction
of the spins lie on the surface, so \emph{corrections} to FSS are quite large
and need to be included in the analysis. In this section, we describe the
method we used to include the \emph{leading} corrections to FSS.

A convenient way to extract the leading scaling behavior from the data, in the
presence of corrections, is the quotient method
\autocite{ballesteros1996finite}, based on the phenomenological scaling of
\textcite{nightingale1976scaling}. As an example, consider the deviation of the
pseudocritical temperature $T_L$ from $T_c$ for which the FSS form is given in
\cref{eq:TL-scaling}. Including the \emph{leading} correction to scaling, which
involves a universal exponent $\omega$, we have
\begin{equation}
  \Delta T(L) \equiv T_c - T_L =
  \frac{A}{L^{\lambda}}\del{1 + \frac{B}{L^{\omega}}}.
  \label{eq:shift}
\end{equation}
We determine the quotient $Q\sbr{\Delta T}$ by taking the logarithm of the
ratio of the result for sizes $L$ and $s L$, where $s$ is a simple rational
fraction such as $2$ or $3/2$,
\begin{equation}
  Q_{s,L}\sbr{\Delta T}
  = \frac{1}{\log s} \log\del{\frac{\Delta T(s L)}{\Delta T(L)}}.
  \label{eq:qshift}
\end{equation}
According to \cref{eq:shift} we have, for large $L$,
\begin{equation}
  Q_{s,L}\sbr{\Delta T} = -\lambda + \frac{C_s}{L^{\omega}}
  \label{eq:qshift-scaling}
\end{equation}
where
\begin{equation}
  C_s = \frac{s^{-\omega} - 1}{\log s} B.
\end{equation}
If the data are of sufficient quality, we can fit all of the unknown
parameters. In \cref{eq:qshift-scaling}, these are the exponents $\lambda$,
$\omega$, and the amplitude $C_s$. In most cases, however, we will need to
assume the predicted value for the correction exponent $\omega$ (see below) to
obtain an unambiguous fit for the remaining parameters.

According to the renormalization group, for $d > d_u = 4$, the leading
irrelevant variable has scaling dimension
\begin{equation}
  \omega = d - 4.
\end{equation}
% TODO: understand this
However, for $k=0$ fluctuations and periodic boundary conditions, it was shown
by \textcite{brezin1985finite} that there is an additional, and larger,
correction for finite-size effects with an exponent given by
\begin{equation}
  \omega^{\prime} = \frac{d-4}{2}.
  \label{eq:correction-k0}
\end{equation}
An intuitive way to see this is to note that the ``naive" variation of $\chi$
with $L$ at the critical point, $\chi \propto L^2$ [see
\cref{eq:fss-above-naive}], although not the dominant contribution [which is
$L^{d/2}$ as shown in \cref{eq:fss-above-susc}], is nonetheless still present
as a correction. This correction is down by a factor of $L^{2-d/2}$
($=L^{-\omega^{\prime}}$) relative to the dominant term. We will therefore use
$\omega^{\prime}$ rather than $\omega$ in considering corrections to scaling
for susceptibilities that scale with $L$ to the power $d/2$ rather than 2.
% TODO: understand this

For some of our data, we will also need subleading corrections to FSS for which
there are several contributions. One of these is the square of the leading
contribution. To avoid having too many fit parameters, this is the form we will
assume, \emph{i.e.}, when we include subleading corrections to scaling we will
do a parabolic fit in $1/L^{\omega}$ (or $1/L^{\omega^{\prime}}$ as the case
may be).

A subtlety arises in doing fits to data for quotients, for example to determine
the parameters $\lambda$, $\omega$, and $C_s$ in \cref{eq:qshift-scaling}. The
reason is that the same set of simulation data may be used to determine more
than one data point in the fit. For example, with $s=2$ the data for $L=16$ is
used in the computation of quotients for pairs (8, 16) and (16, 32).
Furthermore, we will fit the exponents $\lambda$ and $\omega$ simultaneously to
quotients for two different values of $s$ ($s=2$ and 3/2),%
\footnote{%
  This is justified since the exponents are universal. The amplitude $C_s$ is,
  however, nonuniversal, so we include a separate amplitude for each value of
  $s$.
}
so for example we also use the data for $L=16$ to compute the quotient for the
pair (16, 24). This has the advantage of increasing the number of data points
in the fit by more than the number of parameters. However, in this case the
data being fitted are \emph{not statistically independent}, and therefore the
best estimate of the fitting parameters should account for the correlations.
\autocite{ballesteros1996finite,ballesteros1998critical,weigel2009cross}. In
other words, if a data point is $(x_i,y_i)$ and the fitting function is $u(x)$,
which depends on certain fitting parameters, those parameters should be
determined by minimizing
\begin{equation}
  \chi^2 = \sum_{i,j}
  \sbr{y_i - u(x_i)}
  \del{\vec{\Sigma}^{-1}}_{ij}
  \sbr{y_j - u(x_j)},
  \label{eq:chi2-correlated}
\end{equation}
where
\begin{equation}
  \Sigma_{ij} = \av{y_i y_j} - \av{y_i}\av{y_j}
\end{equation}
is the covariance matrix of the data. We determine the elements of the
covariance matrix by a bootstrap analysis (see
\cref{sec:numerical-bootstrap}). If there are substantial correlations, the
covariance matrix can become singular, and where this happens we replace
$\vec{\Sigma}^{-1}$ in \cref{eq:chi2-correlated} with the ``pseudoinverse"
$\vec{\Sigma}^+$.%
\footnote{%
  This corresponds to projecting the covariance matrix onto the eigenvectors
  whose eigenvalues are not (close to) zero and inverting the resulting matrix.
}
The effective number of independent data points is then the rank of the
covariance matrix (the number of non-zero eigenvalues).


\section{Results: periodic boundary conditions}
\label{sec:fss-results-pbc}


\subsection{$k=0$ fluctuations}

Here we show results for completeness, as there is no doubt that the FSS form
of \cref{eq:fss-above} is correct for periodic boundary conditions.

\Cref{fig:binder-p} shows an overview of our data for the Binder ratio $g$,
showing intersections at, or near, the transition temperature $T_c$ given in
\cref{eq:Tc-d5}. The right-hand panel is an expanded view near $T_c$, where it
is clear that intersections for different sizes do not occur at exactly the
same point, indicating corrections to scaling. In fact, the data for smaller
sizes intersect at a value larger than the exact, universal value
\begin{equation}
  g_c =
  \frac{1}{2}\del{3 - \frac{\Gamma^4(\frac{1}{4})}{8\pi^2}} \approx
  0.40578
  \label{eq:binderTc}
\end{equation}
found by \textcite{brezin1985finite}. However, for larger sizes the
intersections occur at smaller values of $g$.
\begin{figure}
  \centering
  \includestandalone{figures/binder-p}
  \includestandalone{figures/binder-p-zoom}
  \caption[
    Data for the Binder ratio $g$ for the five-dimensional Ising model with
    periodic boundary conditions.
  ]
  {
    The left panel shows an overview of our results for the Binder ratio $g$
    for periodic boundary conditions. The right panel is an expanded view near
    the transition. The transition temperature $T_c$ is marked with a
    horizontal line, and the universal value of the Binder ratio at the
    transition temperature, $g_c$, given by \cref{eq:binderTc}, is marked with
    a vertical line.
  } \label{fig:binder-p}
\end{figure}
\Cref{fig:gx} shows estimates of the value of $g$ at $T_c$, plotted against
$L^{-\omega^{\prime}}$ with the correction exponent given by
$\omega^{\prime}=1/2$ as discussed in \cref{sec:fss-quotient}. The data
decrease to a value consistent with \cref{eq:binderTc} as $L\to\infty$.
%The effect of a fairly slow correction to scaling exponent
%$\omega^{\prime}=1/2$, combined with a fairly large correction amplitude, has
%made it difficult to obtain the known exact result for $g_c$ from numerics.
%This should serve as a cautionary tale when applying FSS to other problems
%where the exact answer is not known.
\begin{figure}
  \centering
  \includestandalone{figures/gx}
  \caption[
    Data for the Binder ratio $g$ at the transition temperature $T_c$ for the
    five-dimensional Ising model with periodic boundary conditions.
  ]
  {
    Estimates of the Binder ratio $g$ at $T_c$, plotted against
    $L^{-\omega^{\prime}}$ with $\omega^{\prime}=1/2$, see
    \cref{eq:correction-k0}, and a linear fit indicating an extrapolated
    value for $L\to\infty$ consistent with the exact result, $g_c \approx
    0.406$ (marked with a horizontal line in the figure), see
    \cref{eq:binderTc}. The estimates were obtained from a cubic smoothing
    spline fit to data at and near $T_c$, and the error bars were estimated
    using the bootstrap procedure. The quality of the fit is good, $Q=0.22$.
  } \label{fig:gx}
\end{figure}


\subsection{$k \neq 0$ fluctuations}

The data for $\chi(\vec{k})$ for $\vec{k}L/(2\pi)=(1,0,0,0,0)$ are shown in
\cref{fig:chi-p-k10}. Note that the Fourier components at nonzero wave vector
vanish even in the ordered state below $T_c$, and so what we define as
$\chi(\vec{k})$ really is the susceptibility below $T_c$ as well as above
[unlike the $k=0$ susceptibility defined in \cref{eq:susc-k0}]. Consequently
the data have a peak, whereas the uniform ``susceptibility" plotted in
\cref{fig:data-f-susc} (for free boundary conditions) continues to increase
below $T_c$.

A scaling plot of the data according to standard FSS of
\cref{eq:susc-kn-scaling} is shown in \cref{fig:chi-k10-scaling}. Except for
the smallest size, $L=8$, near $T_c$ the data scale very well. Further from
$T_c$ on the low-$T$ side, we see bigger corrections. However, this is
unsurprising since FSS is only expected to work for $T$ close to $T_c$.

For larger $k$ values we get a similar picture, albeit with bigger corrections
to scaling, as shown in \cref{fig:chi-k110-scaling} for
$\vec{k}L/(2\pi)=(1,1,0,0,0)$. It is expected that corrections to scaling
become \emph{relatively} bigger for larger $k$ because the signal is less
divergent in this case, and so it is more easily affected by corrections.
% TODO: explain

\begin{figure}
  \centering
  \includestandalone{figures/chi-p-k10}
  \caption [
    Wavevector-dependent susceptibility $\chi(\vec{k})$ for $\vec{k} L/(2\pi) =
    (1,0,0,0,0)$ for the five-dimensional Ising model with periodic boundary
    conditions.
  ]
  {
    Susceptibility $\chi(\vec{k})$ for $\vec{k} L = (1,0,0,0,0)$, which we
    abbreviate to $\chi_{10}$, for periodic boundary conditions.
  }
  \label{fig:chi-p-k10}
\end{figure}

\begin{figure}
  \centering
  \begin{subfigure}{0.49\textwidth}
    \centering
    \includestandalone{figures/chi-p-k10-scaling}
    \subcaption{$\vec{k}L/(2\pi)=(1,0,0,0,0)$}
    \label{fig:chi-k10-scaling}
  \end{subfigure}
  \begin{subfigure}{0.49\textwidth}
    \centering
    \includestandalone{figures/chi-p-k110-scaling}
    \subcaption{$\vec{k}L/(2\pi)=(1,1,0,0,0)$}
    \label{fig:chi-k110-scaling}
  \end{subfigure}
  \caption[
    Scaling plots of the susceptibility of the $d=5$ periodic Ising model for
    two nonzero wavevectors.
  ]
  {
    Scaling plots of the susceptibility for two nonzero wavevectors. Panel
    \subref{fig:chi-k10-scaling} shows the scaled data of \cref{fig:chi-p-k10}.
  } \label{fig:chi-p-modes-scaling}
\end{figure}

\Cref{fig:chi-modes-p-Tc} shows the behavior of $\chi(\vec{k})/L^2$ at $T_c$
showing that it is a function of the product $k L$ as expected; see
\cref{eq:susc-kn-scaling}. The dashed line has slope $-2$ indicating that the
expected $k^{-2}$ behavior of \cref{eq:susc-kn-Tc-scaling} sets in even for
small values of $k L$.
\begin{figure}
  \centering
  \includestandalone{figures/chi-modes-p-Tc}
  \caption[Values of $\chi(\vec{k})/L^2$ at $T_c$ for periodic boundary conditions.]
  {
    Values of $\chi(\vec{k})/L^2$ at $T_c$ for periodic boundary conditions.
    Each group of points has the same $x$ coordinate (1, $\sqrt{2}$, or 2), but
    the points are displaced slightly horizontally so that they can be
    distinguished. There are two different wavevectors shown for $k
    L/(2\pi)=2$, namely $\vec{k}L/(2\pi)=(2,0,0,0,0)$ and $(1,1,1,1,0)$. These
    two agree well except for the smaller sizes, showing that the fluctuations
    are isotropic at long wavelength. The dashed line has slope $-2$,
    indicating that the expected $k^{-2}$ behavior in
    \cref{eq:susc-kn-Tc-scaling} sets in even for small values of $k L$.
  } \label{fig:chi-modes-p-Tc}
\end{figure}


\section{Results: free boundary conditions}
\label{sec:fss-results-fbc}

Because corrections to scaling are larger for free boundary conditions than for
periodic boundary conditions, in this section we make extensive use of the
quotient method described in \cref{sec:fss-quotient} to incorporate the
leading corrections to scaling.


\subsection{$k=0$ fluctuations}

An overview of our results for the Binder ratio is shown in
\cref{fig:data-f-binder}.
\begin{figure}
  \centering
  \begin{subfigure}{0.49\textwidth}
    \centering
    \includestandalone{figures/binder-f}
    \subcaption{}\label{fig:data-f-binder}
  \end{subfigure}
  \begin{subfigure}{0.49\textwidth}
    \centering
    \includestandalone{figures/chi-f}
    \subcaption{}\label{fig:data-f-susc}
  \end{subfigure}
  \caption[
    Data for the Binder ratio $g$ and the susceptibility $\chi$ for the
    five-dimensional Ising model with free boundary conditions.
  ]
  {
    Overview of data for \subref{fig:data-f-binder} the Binder ratio $g$ and
    \subref{fig:data-f-susc} the susceptibility $\chi$, for free boundary
    conditions. Note the large shift to lower temperatures for the smaller
    sizes and lack of any apparent intersections of the data for different
    sizes in \subref{fig:data-f-binder}.
  } \label{fig:data-f}
\end{figure}
We do not find any intersections, and the data are shifted considerably to
lower temperatures for smaller sizes.

To determine the shift exponent, we define the pseudocritical temperature $T_L$
to be the temperature at which $g$ takes the value 1/2, halfway between its
limiting values of 0 and 1. We subtract $T_c$, given in \cref{eq:binderTc},
and determine the resulting quotients for $\Delta T(L) \equiv T_c - T_L$
according to \cref{eq:qshift}. We then fit \cref{eq:qshift-scaling}
to the quotients, as shown in \cref{fig:qshift}.
\begin{figure}
  \centering
  \includestandalone{figures/qshift}
  \caption[
    Quotient estimation of the shift exponent $\lambda$ for the
    five-dimensional Ising model with free boundary conditions.
  ]
  {
    Quotients for $\Delta T(L)$, defined in \cref{eq:shift}, used to
    determine the shift exponent $\lambda$ for free boundary conditions. The
    parameter $s$ is the ratio of the two sizes used to compute the quotient.
    We fit \cref{eq:qshift-scaling} to the data using as parameters $\lambda$,
    $\omega$ (the same for both values of $s$), and separate amplitudes
    $C_{3/2}$ and $C_2$. The quality of the fit is very good, $Q=0.96$.
  } \label{fig:qshift}
\end{figure}
The quality of the data is very good and we are able to fit all four
parameters, $\lambda$, $\omega$, and the two amplitudes $C_s$. We find the
values
\begin{equation}
  \lambda = 2.003(3),\quad
  \omega = 1.01(3).
\end{equation}
The value for the shift exponent is in precise agreement with the value
$\lambda=2$ proposed analytically by \textcite{rudnick1985effect} and found
numerically by \textcite{berche2012hyperscaling}. There is also excellent
agreement between our value of the correction exponent $\omega$ and the RG
value of 1.

We estimate the rounding by the range in temperature $\delta T(L)$ over which
$g$ varies between 0.25 and 0.75, \emph{i.e.}
\begin{equation}
  \delta T(L) = T(g=0.25) - T(g=0.75).
  \label{eq:width}
\end{equation}
Computing the quotients and fitting the form
\begin{equation}
  Q_{s,L}\sbr{\delta T} = -y_t^* + A_s/L^{\omega},
  \label{eq:qwidth}
\end{equation}
we find that the data are insufficient to determine the three parameters, but
if we assume the RG value for the correction exponent, $\omega=1$, then we get
a good fit that extrapolates to
\begin{equation}
  y_t^* = 2.50(3),
  \label{eq:fss-above-yt-fit}
\end{equation}
% TODO: update with result of new analysis
see \cref{fig:qwidth}, in precise agreement with the prediction $d/2$, see
\cref{eq:fss-above-yt}.
\begin{figure}
  \centering
  \includestandalone{figures/qwidth}
  \caption[
    Quotient estimation of the width exponent $y_t^*$ for $k=0$ modes of the
    five-dimensional Ising model with free boundary conditions.
  ]
  {
    Quotients for $\delta T(L)$, defined in \cref{eq:width}, used to
    determine the rounding exponent $y_t^*$ for free boundary conditions. We fit
    \cref{eq:qwidth} to the data using as parameters $y_t^*$, (the same
    for both values of $s$) and separate amplitudes. The value of the
    correction exponent is fixed to $\omega=1$.
  } \label{fig:qwidth}
\end{figure}
We note that the quoted error bar \emph{assumes} that the data can be described
by \cref{eq:fss-above-yt-fit}; in other words, that \emph{subleading}
corrections do not affect the fitted data significantly. If this is the case,
we have established the values of the shift and rounding exponents in
\cref{eq:fss-above-shift-binder}.

Now we consider the scaling of $\chi$ in \cref{eq:fss-above-shift-susc}. The
data for $\chi$ are shown in \cref{fig:data-f-susc}. From this we estimate the
value of $\chi$ at $T_L$ (where $T_L$ is determined, as before, from where $g$
takes the value 1/2) and do a quotient analysis, shown in \cref{fig:qchiTL}.
\begin{figure}
  \centering
  \includestandalone{figures/qchiTL}
  \caption[
    Quotients for the value of the susceptibility $\chi$ at the finite-size
    pseudocritical temperature for the five-dimensional Ising model with free
    boundary conditions.
  ]
  {
    Quotients for the value of $\chi$ at $T_L$ for free boundary conditions
    plotted against $L^{-\omega^{\prime}}$, where the correction to scaling
    exponent $\omega^{\prime}$ is fixed to the value $1/2$. According to
    \cref{eq:fss-above-susc}, the quotients should extrapolate to the value
    $y_t^*$ ($=5/2$) for $L\to\infty$. The linear fit omits the right-hand
    point for each of the two data sets. There are three fitting parameters:
    $y_t$ and two amplitudes for the correction, one for each value of $s$. The
    quality of fit is good, $Q=0.21$.
  }
  \label{fig:qchiTL}
\end{figure}
The data are insufficient to determine the correction to the scaling exponent,
so we fixed it to the expected value $\omega^{\prime}=1/2$, see
\cref{eq:correction-k0}. The amplitude of the correction term is large, but the
data extrapolate to a value 2.51(1),
% TODO: update with final value
consistent with the value of $y_t^*=5/2$ expected from
\cref{eq:fss-above-shift-susc}, and which was found in earlier simulations by
\textcite{berche2012hyperscaling,kenna2013new}.

We also measure $\chi$ at the bulk $T_c$. As shown in \cref{eq:rounding-k0-Tc},
this is proportional to $L^2$, not $L^{d/2}$, and so, as discussed in
\cref{sec:fss-quotient}, we expect that the correction to scaling exponent will
be $\omega$ ($=1$) rather than $\omega^{\prime}$ ($=1/2$). Quotients of the
results are plotted in
\cref{fig:qchiTc}.
\begin{figure}
  \centering
  \includestandalone{figures/qchiTc}
  \caption[
    Quotient estimation of the width exponent $y_t$ for $\vec{k}\neq\vec{0}$
    modes of the five-dimensional Ising model with free boundary conditions.
  ]
  {
    Quadratic fit to the quotients for the value of $\chi$ at the bulk $T_c$
    for free boundary conditions against $1/L^{\omega}$ where the correction
    to scaling exponent $\omega$ is fixed to the value 1.
    According to \cref{eq:rounding-k0-Tc}, the quotients should extrapolate
    to the value of $y_t$ ($=2$). There are five fitting parameters:
    $y_t$ and the amplitudes of the linear and quadratic corrections for
    each $s$ value. The quality of fit is good, $Q=0.43$.
    % TODO: update with latest value
  }
  \label{fig:qchiTc}
\end{figure}
There are clearly subleading corrections to scaling, so we use a quadratic fit.
The result, $y_t=1.97(6)$,
% TODO: update
agrees well with the expected value of 2. We note that the corrections to
scaling are quite large, which is not surprising since the values of $\chi$ at
$T_c$ are small, and thus are more influenced by corrections to scaling than
the data at $T_L$, where $\chi$ is larger.
% TODO: why?
We also tried a linear fit omitting the smallest size for each value of $s$,
finding 1.89(2) with $Q=0.3$, which differs by more than the error bar from the
value 2. Nonetheless, the quadratic fit shows that, although we have not
determined the exponent with which $\chi$ diverges at $T_c$ with great
accuracy, our result is at least \emph{consistent} with the value of 2 expected
according to \cref{eq:rounding-k0-Tc}. An $L^2$ divergence in the
susceptibility has also been found recently by \textcite{lundow2014finite}, who
were able to study larger sizes than those studied here, up to $L=160$.
% TODO: how is L=160 possible?

\Cref{fig:chi-f-scaling} shows a scaling plot of $\chi(T)/\chi(T_L)$ against
$L^{d/2}(T-T_L)/T_L$.
\begin{figure}
  \centering
  \includestandalone{figures/chi-f-scaling}
  \caption[
    Scaling plot of the susceptibility $\chi$ for the five-dimensional Ising
    model with free boundary conditions.
  ]
  {
    Scaling plot of the data for $\chi$ for free boundary conditions according
    to \cref{eq:fss-above-susc}. Also shown are the data at $T_c$, which
    are seen to lie on the scaling function (within some small corrections).
  }
  \label{fig:chi-f-scaling}
\end{figure}
We have seen in \cref{fig:qchiTL} that there are corrections to the expected
$L^{d/2}$ behavior of $\chi$ at $T_L$ for the range of sizes studied. Thus we
divide $\chi(T)$ by $\chi(T_L)$ rather than by $L^{d/2}$, which appears in
\cref{eq:fss-above-shift-susc}, to eliminate the corrections seen in
\cref{fig:qchiTL}. According to \cref{eq:fss-above-shift-susc}, the data in
\cref{fig:chi-f-scaling} should collapse. There are some corrections to this,
which is not surprising since we are probing the scaling function over a large
region, but overall the data scale fairly well. Also shown are data at $T_c$,
which appear at different points for different sizes because $T_L$ is, of
course, size-dependent. The larger the size, the further to the right is the
data point for $T_c$. This figure supports our claim that the data at $T_c$ are
included in the scaling function in \cref{eq:fss-above-shift-susc}.

We have defined the pseudocritical temperatures $T_L$ and the resulting shift
exponent $\lambda$ in \cref{eq:shift} by the temperature where the Binder ratio
takes the value 1/2. Suppose we took a different criterion for $T_L$, such as
the temperature at which the Binder ratio has some other value, or where there
is a peak in some $\vec{k}\neq\vec{0}$ susceptibility, such as that shown in
\cref{fig:chi-p-k10}. We note that the finite-size width varies as $1/L^{d/2}$,
so temperatures at which the Binder ratio has a value between 0 and 1 would lie
in this range, and so they would only give a \emph{subleading} contribution to
the shift, the coefficient of $1/L^2$ remaining the same. We expect that the
\emph{same} shift amplitude would be obtained no matter what quantity is used
to define the shift for the following reason. Suppose we have a shift amplitude
$A$ and pseudocritical temperatures $T_L$ determined from where the Binder
ratio is 1/2 and a different amplitude $A^{\prime}$, and correspondingly
different temperatures $T_L^{\prime}$, determined by some other criteria. Then
the Binder ratio has a scaling form in \cref{eq:fss-above-shift-binder}, but if
we try to define it in terms of the alternative shift temperatures
$T_L^{\prime}$, we have
\begin{align}
  g(L,T)
  &= \scalefunc{g}\sbr{L^{d/2}\del{T-T_L}} \\
  &= \scalefunc{g}\sbr{L^{d/2}\del{T-T_L^{\prime}} +
     \del{A^{\prime}-A}L^{d/2-2}}.
\end{align}
Thus, if different quantities have different shift amplitudes, the argument of
the scaling function would be shifted by an \emph{infinite} amount (for
$L\to\infty$) if we use the shift obtained from a different quantity, a clear
violation of scaling. We therefore postulate that this does not happen and that
there is a \emph{unique} shift amplitude for a given system.

Note, however, that we cannot rule out subleading corrections to the shift of
order $1/L^{d/2}$. As a result, the value of $g$ at $T_L$ according to
\cref{eq:fss-above-shift-binder} will depend on the precise definition of $T_L$
and therefore will \emph{not} be universal, unlike the situation with periodic
boundary conditions; see \cref{eq:fss-above-binder}. Thus one can view the
replacement of \cref{eq:fss-above} by \cref{eq:fss-above-shift} as a violoation
of the standard finite-size scaling \autocite{rudnick1985effect}.
% TODO: why?
However, since the behavior of $\chi$, for example, is described by a single
function both at $T_c$ and $T_L$, we view \cref{eq:fss-above-shift} as
representing a \emph{modified FSS}, distinct from standard FSS, in that it has
different shift and scaling exponents.


\subsection{$k \neq 0$ fluctuations}

With free boundary conditions, the Fourier modes are sine waves given by
\cref{eq:allowed-free}. Modes in which all the integers $n_{\alpha}$ are odd
have a projection onto the uniform magnetization (\emph{i.e.} the $k=0$ mode)
and therefore will acquire a nonzero magnetization in the ordered phase. Such
modes will therefore be affected by the dangerous irrelevant variable, and so
have the same scaling as fluctuations of the uniform magnetization, given in
\cref{eq:fss-above-shift-susc}. We therefore take the smallest wave vector with
an even $n_{\alpha}$, namely $\vec{n}=(2,1,1,1,1)$, since this will not acquire
a nonzero magnetization, so we expect it to be governed by the FSS in
\cref{eq:susc-kn-scaling}, \emph{i.e.}, with exponent 2 rather than $d/2$ which
appears in \cref{eq:fss-above-shift-susc}. We present the data in
\cref{fig:chi-f-k21}.
\begin{figure}
  \centering
  \includestandalone{figures/chi-f-k21}
  \caption[
    Wavevector-dependent susceptibility $\chi(\vec{k})$ for
    $(L+1)\vec{k}/\pi=(2,1,1,1,1)$ for the five-dimensional Ising model with
    free boundary conditions.
  ]
  {
    Data for $\chi(\vec{k})$ for $(L+1)\vec{k}/\pi=(2,1,1,1,1)$ for free
    boundary conditions.
  }
  \label{fig:chi-f-k21}
\end{figure}

According to \cref{eq:susc-kn-scaling}, the height of the peaks in
\cref{fig:chi-f-k21} should scale as $L^2$ and the width should scale as
$L^{-2}$. We define the width to be the difference between the two temperatures
where the susceptibility is 3/4 of the maximum. Quotient analyses for the
height and width are shown in \cref{fig:quots-k21}. For the height, the
(quadratic) fit gives an extrapolated value of 2.008(10), consistent with the
expected value of $y_t=2$ As discussed in the caption of \cref{fig:qchiTL-k21},
a linear fit gave a value 1.950(2), close but slightly different from 2.
However, in this case the quality of fit $Q=0.02$ was unacceptably low, which
is why we went to a quadratic fit. For the data of the width in
\cref{fig:qwidth-k21}, the amplitudes of the corrections are small and we find
an extrapolated value of $-1.97(4)$, consistent with the expected value of
$-y_t$ ($=-2$).
\begin{figure}
  \centering
  \begin{subfigure}{0.49\textwidth}
    \centering
    \includestandalone{figures/qchiTL-k21}
    \subcaption{}\label{fig:qchiTL-k21}
  \end{subfigure}
  \begin{subfigure}{0.49\textwidth}
    \centering
    \includestandalone{figures/qwidth-k21}
    \subcaption{}\label{fig:qwidth-k21}
  \end{subfigure}
  \caption[
    Quotient estimates of the rounding exponent $y_t$ for $\vec{k}\neq\vec{0}$
    modes of the five-dimensional Ising model with free boundary conditions.
  ]
  {
    Quotients for the height \subref{fig:qchiTL-k21} and width
    \subref{fig:qwidth-k21} of the peak in $\chi(\vec{k})$ for
    $(L+1)\vec{k}/\pi=(2,1,1,1,1)$ for free boundary conditions. According to
    \cref{eq:susc-kn-scaling}, the quotients of the peak height should
    tend to the value $y_t$ ($=2$) and the quotients for the width should tend
    to $-y_t$ ($=-2$) as $L\to\infty$. The estimates of $y_t$ obtained by
    extrapolation for both fits are consistent with 2. For both fits we fix the
    value of the correction exponent to $\omega=1$. In \subref{fig:qchiTL-k21}
    the correction amplitude is large, but the data are of good quality and a
    quadratic fit works well, $Q=0.53$. A linear fit to these data gave an
    extrapolated value of 1.950(2) but with a poor quality of fit, $Q=0.02$. In
    \subref{fig:qwidth-k21}, the amplitude of the leading correction is seen to
    be quite small, and we use a linear fit which works well, $Q=0.67$.
    % TODO: update values
  }
  \label{fig:quots-k21}
\end{figure}

Thus, we have found strong evidence to suppert our claim that
\cref{eq:susc-kn-scaling} applies to free boundary conditions. Note that
since this FSS form uses $y_t$ ($=2$) and the deviation of $T_L$ from $T_c$ is
proportional to $1/L^2$, asymptotically we can use eigher $T_c$ or $T_L$ in
\cref{eq:susc-kn-scaling}.


\section{Conclusions}

Our main conclusions have already been discussed in \cref{sec:fss-intro}, so
here we summarize our main results:

(i) The \emph{modified} FSS form with exponents $d/2$ rather than 2 only
applies to $k=0$ fluctuations. (For free boundaries it applies to Fourier modes
that have a projection onto the uniform magnetization.) For all other
wavevectors, standard FSS with an exponent 2 applies. Consequently, the
exponent $\eta$ describing the power-law decay of correlations at $T_c$ is
unambiguously $\eta=0$.
(See \cref{fig:chi-p-modes-scaling,fig:chi-modes-p-Tc,fig:quots-k21}.)

(ii) For free boundary conditions and $k=0$, the shift, with an exponent 2, is
larger than the rounding, which has an exponent $d/2$. Using $T-T_L$, where
$T_L$ is the finite-size pseudocritical temperature, rather than $T-T_c$ as a
scaling variable, the data have a scalng form that incorporates both the
behavior at $T_L$, where $\chi \propto L^{d/2}$, and at the bulk $T_c$, where
$\chi \propto L^2$.
(See \cref{fig:qshift,fig:qwidth,fig:qchiTL,fig:qchiTc,fig:chi-f-scaling}.)



\label{sec:fss-conclusions}
